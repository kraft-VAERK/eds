\section{Ucloud - Infratructure as a Service (IaaS)}
\label{sec:ucloud}
The project's goal is to shed light on the security risks associated with using pipelines. 
Most IaaS providers, such as GCP, AWS, Azure, and others, entail significant costs. 
However, SDU is a partner in the recently launched Ucloud project, offering educators, professors, 
and students like myself affordable access to real computing power, potentially provided as a grant for research projects.

The operation of Ucloud differs somewhat from other IaaS providers. 
Despite being an IaaS provider, I'd like to provide an overview of IaaS, emphasizing its characteristics. 
Subsequently, I will delve into the unique aspects of how Ucloud functions and outline the distinctions 
between Ucloud and other IaaS providers.


\subsection{IaaS}
\label{subsec:iaas}
IaaS refers to a cloud computing service wherein the provider hosts 
infrastructure components typically found in an on-premises data center. 
These components include servers, storage, networking hardware, and the virtualization or hypervisor layer. 
The provider also offers various accompanying services, such as detailed billing, monitoring, log access, security features,
load balancing, clustering, and storage resiliency, encompassing backup, replication, and recovery (Source: \cite{iaas}).

Ucloud bears some resemblance to a conventional IaaS provider but exhibits distinct characteristics. 
The primary difference lies in Ucloud's focus on the university environment, 
deviating from a public-oriented approach and the development of traditional applications. 
Instead, Ucloud is designed for the utilization of computational power, akin to a regular IaaS, 
with a specific emphasis on research and education.

Given its tailored use for application development, 
Ucloud does not necessitate certain security measures that would be essential for a more public-facing IaaS.

\section{Ucloud and cloud infrastructure.}
\label{sec:ucloud}
For the project I decided that using the internal available cloud product ucloud would be sufficient, 
to deploy the pipeline. Thinking that Ucloud would be able to supply or atleast help me with the request I would 
have to create an infrastructure which relied on components spawned in docker images.

The first problem I encountered with Ucloud\cite{ucloud} was the allocating of funds. For me to able to have access to the 
computational power of ucloud they will have to grant me funds to be able to run their products. I dont have a problem with 
the need of funds to run machines, its the allocation process for student like me that is questionable at best.
When it comes to use of these machines things become very bureaucratical.
I had to start with already some funds that were allocated to me for being a \ac{CS} student at University of Southern Denmark.


The next problem was the nested virulization. Nested virtulization is important because, Ucloud will spawn 
a virtual machine, on there bare metal for me which i then get granted access to. Since I want to run docker images
i need to have access to the PID 1 on the given machines to have access the linux systemctl. If i don't
have access to PID 1, I wont be able to start the docker daemon and therefore not be able to run docker images.

Upon actualy being able to start working with Ucloud, the task of creating an infrastructure and using it publicly, was not as easy as I thought.
The first problem was that Ucloud does not have a product where i can upload my own docker images, and then deploy them.
I need to spawn machines and then run the docker image. Although this is not a huge problem, because actually 
having access to the machines that runs the docker images is actually kinda nice.

\subsection{Deployment and connection to Ucloud}
\label{subsec:deployment-and-connetion-ucloud}
\subsubsection{Deployment}
\label{subsubsec:deployment-ucloud}

Ucloud has different section of where you can deploy virtual machines. They are located at different universities around Denmark.
Those that are being used are mostly located at SDU and AAU. SDU instances has a single issues that makes that made them inenligible for
use and it's that they dont support nested virtualization. This is a problem, since the use of docker for the deployment of 
gitea and in general for the project is crucial, but more on docker in \ref{docker}

The AAU machines are spawned through the platform that ucloud supplies at \url{https://ucloud.aau.dk/}. How you spawn are machines is done 
in a couple of steps, first you need to figure what OS and version you want of that. Next name your instance and what type of 
virtual hardware you want. What was mostly used during the development of this project was 4 vCPU, 16GB RAM and 100GB disk.
The last thing before spawning the instance is supply an SSH-key, which is used to connect to the instance.

\subsubsection{Connection}
\label{subsubsec:connection-ucloud}
The connection to the instance is done through SSH. 
The SSH-key that was supplied during the spawning of the instance is used to connect to the instance. One of the main issues of the 
method of connetion is that there is no real possible for me to utilize the SSH connetion in a clever way. It is going to be done through 
port forwarding between multiple machines located inside the AAU-cluster network, and even then it's not possible for the machines 
to acutal be exposed to the internet as a web server. Because this is happening is becuase ucloud infrastructure is located behind a 
firewall, which is in front of the network called The Danish Research Network (Forskningsnettet). This is a network that is
used by all the universities in Denmark, and is used to connect them to the internet. The firewall is used to protect the
infrastructure from attacks from the internet. For this project, that is a problem, since the Gitea server will have to be publicly available 
for user to connect to it. 
The solution to this problem was solved by using Ngrok.

\subsection{Ngrok}
\label{subsec:ngrok}


