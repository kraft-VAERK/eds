\section{Gitea - Git Server}

\subsection{What is Gitea?}

Gitea\cite{giteadocs} is a self-hosted \ac{Git}\cite{git} service. Similar to GitHub\cite{GitHub}, Gitea allows the user the ability to customize and run 
their own Git server. Gitea has a lot of great features such as: Code hosting meaning you can host your own code on your own server.
Code review, meaning that you are the developer will be able to review code before it is merged into the master branch.\\
\ac{CI/CD}, having the ability to \ac{CI/CD} on your own server.
Its open source, meaning that you'll be able to customize it to your liking and almost everything is 
configurable.

\subsection{Why Gitea?}
Gitea for this project was a given choice. It holds all the keys for what the project demands.
Firstly, it's open sourece, this means that I can customize it to fit the CTF i want and if i face problems, 
there is most likely a way to circumvent it. Secondly, it has \ac{CI/CD} support, this means that the user 
can have a external pipeline software such as Jenkins\cite{jenkinsio} or GitLab CI\cite{gitlab} to run tests on the code before it is
merged into the master branch. Thirdly, it has code review. Now why would code review be important for a \ac{CTF} project?
Will, since the pipeline can be configured in so many ways, and the runners for the pipeline will most likely execute code, 
with either all privileges or at least with some more than a normal developer has. This means that might be some way 
for a malicious user to exploit the pipeline and gain access to runner based on just asking for code reviews.


\subsection{Gitea static installation}
So for this project, having a static installation of Gitea, or rather an installation that is aimed towards 
docker\cite{docker} deployment. That means, that the installation need to be a single image that the user can pull from 
docker hub and run with minimal or no configuration.\\
This means that the image needs to be build with a couple of predifined settings. 
\begin{itemize}
    \item First: Becuase of Gitea is an open source project and the source code is subjective to change at any time.
    To circumvent this, the base image of Gitea is chosen to be version 1.16.5. Now why this version? After some testing and
    observation of different project using a version earlier than 1.19.0 and the reading through the cicd-goat project\cite{cicd-goat}. 1.16.5 which is 
    also used in \cite{cicd-goat} is the version with the most admin options.
    This means that even though im using a long time on developing this project, the base image stay the same, and i 
    can ensure that the end result will be same as when i started the project, and when the user pulls the image.
    \item Second: The configuration of Gitea needs to be static. When Gitea is installed, it will prompt the installer of Gitea 
    to specify what that person wants. This information will be stored in a special file called app.ini. This file is 
    used by Gitea to configure itself\ref{fig:gitea_app_ini}. The rest of the app.ini file for this project
    can be found the source code of the project under Gitea/app.ini.
    \item Third: Users needs to be created at initialization, so that when the user start the image via docker-compose. 
    The database is already populated with users, and the user can start using the service right away.
    This is done with python3\cite{python} script which 
    post users to the database via a \ac{RESTAPI}, with admin credentials.
\end{itemize}

\begin{figure}
    \begin{center}
        \includegraphics[scale=.5]{images/Gitea-appin.JPG}
        \caption{Gitea app.ini}
        \label{fig:gitea_app_ini}
    \end{center}
\end{figure}

After these three steps are done, and they should be done without the user knowing anything about it, the user should be able to
start the image and have a fully functional Gitea server to interact with.

\subsection{Gitea remote}
For the final project, the Gitea repository resides on a remote server. 
This enables users to interact with the server from any location with internet access. Importantly, 
even if the Gitea server goes offline, it can be swiftly restarted from scratch. 
All necessary source code for instant restoration is available, eliminating the need for additional configurations.