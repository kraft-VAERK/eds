\chapter{Discussion}
\label{sec:discussion}
As the need for pipeline security and automation grows, so does the demand
for educational and practice materials in these areas. While there may be
resources available for specific tools, there will always be a need for materials
or platforms that allow for practice in real-world scenarios. Therefore,
having the pipeline available on a local setup and having it available on \ac{HAAUKINS} 
will make the material for educational purposes more accessible.

\section{Ucloud}
\label{sec:discussion-ucloud}
\section{Ucloud - Infratructure as a Service (IaaS)}
\label{sec:ucloud}
The project's goal is to shed light on the security risks associated with using pipelines. 
Most IaaS providers, such as GCP, AWS, Azure, and others, entail significant costs. 
However, SDU is a partner in the recently launched Ucloud project, offering educators, professors, 
and students like myself affordable access to real computing power, potentially provided as a grant for research projects.

The operation of Ucloud differs somewhat from other IaaS providers. 
Despite being an IaaS provider, I'd like to provide an overview of IaaS, emphasizing its characteristics. 
Subsequently, I will delve into the unique aspects of how Ucloud functions and outline the distinctions 
between Ucloud and other IaaS providers.


\subsection{IaaS}
\label{subsec:iaas}
IaaS refers to a cloud computing service wherein the provider hosts 
infrastructure components typically found in an on-premises data center. 
These components include servers, storage, networking hardware, and the virtualization or hypervisor layer. 
The provider also offers various accompanying services, such as detailed billing, monitoring, log access, security features,
load balancing, clustering, and storage resiliency, encompassing backup, replication, and recovery (Source: \cite{iaas}).

Ucloud bears some resemblance to a conventional IaaS provider but exhibits distinct characteristics. 
The primary difference lies in Ucloud's focus on the university environment, 
deviating from a public-oriented approach and the development of traditional applications. 
Instead, Ucloud is designed for the utilization of computational power, akin to a regular IaaS, 
with a specific emphasis on research and education.

Given its tailored use for application development, 
Ucloud does not necessitate certain security measures that would be essential for a more public-facing IaaS.

\section{Ucloud and cloud infrastructure.}
\label{sec:ucloud}
For the project I decided that using the internal available cloud product ucloud would be sufficient, 
to deploy the pipeline. Thinking that Ucloud would be able to supply or atleast help me with the request I would 
have to create an infrastructure which relied on components spawned in docker images.

The first problem I encountered with Ucloud\cite{ucloud} was the allocating of funds. For me to able to have access to the 
computational power of ucloud they will have to grant me funds to be able to run their products. I dont have a problem with 
the need of funds to run machines, its the allocation process for student like me that is questionable at best.
When it comes to use of these machines things become very bureaucratical.
I had to start with already some funds that were allocated to me for being a \ac{CS} student at University of Southern Denmark.


The next problem was the nested virulization. Nested virtulization is important because, Ucloud will spawn 
a virtual machine, on there bare metal for me which i then get granted access to. Since I want to run docker images
i need to have access to the PID 1 on the given machines to have access the linux systemctl. If i don't
have access to PID 1, I wont be able to start the docker daemon and therefore not be able to run docker images.

Upon actualy being able to start working with Ucloud, the task of creating an infrastructure and using it publicly, was not as easy as I thought.
The first problem was that Ucloud does not have a product where i can upload my own docker images, and then deploy them.
I need to spawn machines and then run the docker image. Although this is not a huge problem, because actually 
having access to the machines that runs the docker images is actually kinda nice.

\subsection{Deployment and connection to Ucloud}
\label{subsec:deployment-and-connetion-ucloud}
\subsubsection{Deployment}
\label{subsubsec:deployment-ucloud}

Ucloud has different section of where you can deploy virtual machines. They are located at different universities around Denmark.
Those that are being used are mostly located at SDU and AAU. SDU instances has a single issues that makes that made them inenligible for
use and it's that they dont support nested virtualization. This is a problem, since the use of docker for the deployment of 
gitea and in general for the project is crucial, but more on docker in \ref{docker}

The AAU machines are spawned through the platform that ucloud supplies at \url{https://ucloud.aau.dk/}. How you spawn are machines is done 
in a couple of steps, first you need to figure what OS and version you want of that. Next name your instance and what type of 
virtual hardware you want. What was mostly used during the development of this project was 4 vCPU, 16GB RAM and 100GB disk.
The last thing before spawning the instance is supply an SSH-key, which is used to connect to the instance.

\subsubsection{Connection}
\label{subsubsec:connection-ucloud}
The connection to the instance is done through SSH. 
The SSH-key that was supplied during the spawning of the instance is used to connect to the instance. One of the main issues of the 
method of connetion is that there is no real possible for me to utilize the SSH connetion in a clever way. It is going to be done through 
port forwarding between multiple machines located inside the AAU-cluster network, and even then it's not possible for the machines 
to acutal be exposed to the internet as a web server. Because this is happening is becuase ucloud infrastructure is located behind a 
firewall, which is in front of the network called The Danish Research Network (Forskningsnettet). This is a network that is
used by all the universities in Denmark, and is used to connect them to the internet. The firewall is used to protect the
infrastructure from attacks from the internet. For this project, that is a problem, since the Gitea server will have to be publicly available 
for user to connect to it. 
The solution to this problem was solved by using Ngrok.

\subsection{Ngrok}
\label{subsec:ngrok}




\section{HAAUKINS}
\label{sec:discussion-haaukins}
As discussed in Section \ref{sec:haaukins}, the absence of documentation and general feedback from the \ac{HAAUKINS} team
made it difficult to fully understand \ac{HAAUKINS} and its capabilities.
The intended use of \ac{HAAUKINS} in this project was to manage the infrastructure for the pipeline backend machines and handle flags.

Discussing the errors and issues that \ac{HAAUKINS} encounters during use is challenging because a 
thorough examination of \ac{HAAUKINS}, its features and structure was not been possible.
Consequently, conclusions about \ac{HAAUKINS} will be based on the information obtained 
from its minimal documentation, the GameSS project, and \ac{DDC}.\\\\
The future use of the \ac{HAAUKINS} program within the pipeline involves deploying 
it on an existing \ac{HAAUKINS} instance through the GameSS project. 
From there, CTFs will be integrated into the new pipeline.

Although the pipeline is not entirely new, it requires a fresh configuration for the \ac{HAAUKINS} platform to correctly spawn the backend containers, 
effectively making it a new pipeline. Additionally, because the \ac{HAAUKINS} platform uses specific DNS records, such as \javaf{hkn}, 
the certificate that enables \ac{HTTPS} communication between containers must be recreated to match the specific \javaf{.hkn} \ac{CN}.

\section{Pipeline}
\label{sec:discussion-pipeline}
Two pipelines using the same programs were developed during this project.
One pipeline with a local Drone instance and a remotely connected Gitea server.
The other is a totally localized deployable pipeline with a local Gitea server and a local Drone instance.
For structural purposes, the pipelines are named \javaf{local} and \javaf{remote}.

\subsection{Local pipeline}
\paragraph{Configuration at initialization}
The local pipeline resolved all the issues encountered with the remote pipeline, 
but it required creating several configuration files for initialization. To ensure the pipeline deployment was fully automated, 
all infrastructure components such as users, repositories, webhooks, and secrets had to be set up before the first login to Gitea and Drone.
Since the setup had to be completed before the actual Gitea and Drone servers were running, manual insertion into the database was necessary. 
This meant that for creating secrets, users, webhooks, and other configurations, 
it was essential to insert the data into the databases of both programs. 
This process often led to large SQLite databases queries with incorrect data entries and unpredictable program behavior.\\
Testing the extensive configuration required running the pipeline and all related programs in Docker containers. 
Each configuration and deployment cycle took four to six minutes. Given that the deployment process for the 
localized pipeline was executed around 300-500 times.
\paragraph{Resource consumption}
As the pipeline used Docker to spawn runners inside a Docker image, the resource consumption 
could potentially be high at a point. Drone gives a restriction in how many runners a pipeline or project 
can run a the same time, but there is no restriction in how many Docker containers a runners can spawn to be able to 
complete its execution. This could potentially lead to a high resource consumption on the host machine.
To prevent this a quota for the runners could be set, but this was not implemented in the pipeline.
Because the quota wasn't implemented into the runners was because a thorough testing of computational resources 
and how much a runner would consume was not done during extreme load. However, it is uncertain whether, if implemented, 
this features should be something done in \ac{HAAUKINS} or something done directly into the pipeline. Implementing
would also mean for local deployment that overusing resources would not be a problem.

\subsection{Remote pipeline}
\paragraph{OAuth2 issue}
The remote pipeline was constructed to address an issue encountered with OAuth2. 
The problem arose because the Drone server and the Gitea server were both running on the same machine using the same URL, 
localhost, differentiated only by port numbers. This caused OAuth2 to malfunction, as it could not distinguish between the two servers, 
resulting in redirects to the wrong server and failed authentication attempts.\\
Through extensive testing and debugging, the solution was found to separate the two servers and assign a 
different URL to the Gitea server. This was achieved by deploying the Gitea server on a remote machine while keeping the drone server 
on a local machine. This allowed OAuth2 to differentiate between the two servers, leading to a successful authentication process.\\
Later in the thesis process, it was discovered that this 
issue could also be resolved on the same machine by using different URLs for the two servers and placing them behind a reverse proxy.
\paragraph{Pipeline structure}
In the remote pipeline setup, the Drone server was localized while the Gitea server was remote. 
This arrangement made the connection and communication between the two servers challenging. 
Normally, both the Drone and Gitea servers would be exposed to the internet. 
However, automating the exposure of the Drone server to the internet was not resolved.\\
To facilitate communication between the two servers, an SSH connection from the Gitea server machine 
to the Drone server was used. This allowed the Gitea server to send \javaf{GET, POST, DELETE, and PUT} requests through 
the SSH connection to the Drone server. This approach was necessary because the Gitea server resided in a VM on \ac{Ucloud}, 
and, as mentioned in section \ref{sec:connection-ucloud}, the VMs on \ac{Ucloud} are only accessible via SSH.
After the \javaf{drone.sh} script was completed, users had to manually create a key pair for the Gitea server, 
enabling it to establish an SSH connection and forward traffic to the Drone server.

Having this structure for the pipeline was not optimal. It had many 
drawbacks and wasn't really an automated deployment. There was almost issues with either 
creating users, webhooks, push repositories to remote Gitea, pipeline not being able to sync with the Gitea server,
Gitea not being able to communicate with the Drone server and so on.
\paragraph{Ngrok issue}
For having the Gitea remotely accessible Ngrok was used to create a public \ac{URL} to the 
server and exposing it. Although Ngrok offers free usage with a single domain, a change in the Terms of Service (TOS) 
limited the amount of data available to free users per month. This limitation rendered the pipeline unusable once 
the data quota was exhausted, making the remote Gitea server inaccessible. 
Consequently, Ngrok was abandoned, and the pipeline was moved to a localized solution.



\section{Reflections and improvements towards CTF's}
\label{sec:discussion-future-work}
Although the project is complete, the discussion has highlighted several problems and challenges encountered during its development. 
This reflection has made it clear that there are still areas that need improvement and further work in the future. 

\subsection{Last CTF, Supply chain attack}
As mentioned at the end of section \ref{sec:ctf}, the last CTF in development 
was not finished due to time constrains.
This CTF is likely one of the more challenging ones, both in terms of implementation and for the user to solve. 
Completing this CTF would have been a valuable addition to the pipeline.
Although some of the development aspects of the creating CTF's was challenging,
there were two main challenges that were unable to be achieved.
\paragraph{Cron Job, for timed execution}
Whenever creating a \javaf{Cron Job} for Drone the software, whenever creating it through the API or the CLI,
adds the next execution time based on whenever the \javaf{Cron Job} was created.\\
However, since the \javaf{Cron Job} was created by inserting it directly into the database, 
the software was not present to do next execution time calculation and therefore the insertion of a wrong format in the database makes the 
UI break for the \javaf{Cron Job} when viewed on the Drone UI.
\paragraph{Implementation a exploit through the Docker image}
The proposed solution for the \ac{CTF} challenge was to modify the Docker image used by Ulduar in its pipeline. 
This Docker image could be updated through another repository Icecrown, 
which had a pipeline containing credentials capable of altering the Docker image in the registry.
However, during the development of the \ac{CTF}s, 
it became clear that the Drone pipeline removes any commands or environmental variables inserted into the Docker image it uses. 
Consequently, without knowing if there is an exploit of this nature for the Drone pipeline, implementing an exploit for this \ac{CTF} proved difficult.

Therefore in the end, it was not achieved to implement this \ac{CTF} in the pipeline. 
The source code for the \ac{CTF} is available in the repository, and it is possible to implement it in the future.

\subsection{\ac{CTF}s on the pipeline}
In the end of the thesis, it was possible for the pipeline to be tested by real users. 
As for most of the project, it was only the author that tested, the feedback from the users was valuable.
Here is some of the feedback given, but all the feedback is included in the appendix \ref{app:feedback}.
\begin{itemize}
    \item \textbf{Amirdrassil} 
    One of the users had a difficult time, figuring out where the flag was located, because there 
    was not any mention or use of any kind of security aspect in the pipeline. The user 
    expressed that it would be an idea to change the password for one of the services used in the pipeline, to 
    use the an environmental variable which was the flag.
    \item \textbf{Firelands}
    Having a file named "flag.py" makes every user just go to that file and grab the flag. 
    The flag should be hidden better. The feedback gives a general sense of the training challenges missed its mark, 
    because the user can just grab the flag without solving the challenge.
    \item Another mentioned that instead of just handing over the credentials to the user account 
    on Gitea and Drone, the user should have to figure out how to get the credentials.
\end{itemize}

The feedback indicates that the challenges in the pipeline does work, but when gamifying
it, is important that there is only one way to solve challenges. When users does \ac{CTF}, they 
will always choose the fastest option to exploit or solve the challenge.