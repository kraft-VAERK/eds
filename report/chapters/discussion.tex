\chapter{Discussion}
\label{sec:discussion}
As the need for pipeline security and automation grows, so does the demand
for educational and practice materials in these areas. While there may be
resources available for specific tools, there will always be a need for materials
or platforms that allow for practice in real-world scenarios. Therefore,
having the pipeline available on a local setup and having it available on \ac{HAAUKINS} 
will make the material for educational purposes more accessible.

\section{Ucloud}
\label{sec:discussion-ucloud}
\subsection{\ac{Ucloud} and cloud infrastructure.}
For the pipeline to run together with \ac{HAAUKINS}, a kind of computational power or cloud provider is necessary.
As \ac{SDU} is part of the \ac{Ucloud}\cite{ucloud}, a computational cloud provider, this thesis explored the 
use of \ac{Ucloud} to deploy \ac{HAAUKINS} and the pipeline on that infrastructure.

One of the first problems encountered with \ac{Ucloud} was the allocating of funds. In order 
to spawn and run the desired machines necessary for a project on \ac{Ucloud}, ultimately these machines would have \ac{HAAUKINS} and 
the pipeline running on them.
Applying for grants is something that is done by a professor or a researcher at \ac{SDU}, and as a student,
I had no influence over the allocation of funds.

The next problem was the nested virtualization. Nested virtualization is 
when a operating system gives the user or itself the possibility to run a virtual machine inside a virtual machine. As the 
name suggests, it is a virtualization inside a virtualization.
Nested virtualization is important because, \ac{Ucloud} will spawn 
a virtual machine, on their bare metal which they grant the user access to. Since this project heavily focuses on running Docker images, 
it is crucial to have access to PID 1 on the given machines in order to utilize the Linux systemctl. If the machine is booted without PID 1, 
which is essentially the Linux init system, it will not be able to run Docker images.

Upon starting to work with \ac{Ucloud}, the task of creating an infrastructure and using it publicly, was not as easy as initially thought.
One of the core offerings of an \ac{IaaS} is the ability to run your code and applications on their infrastructure. 
However, Ucloud does not allow users to spawn their own Docker images directly on their infrastructure. Instead, to run Docker images, 
a user must launch an application defined on the Ucloud infrastructure 
and then run the Docker images on that machine, provided the machine has access to PID 1.

Ultimately, the project was completed without using \ac{Ucloud}, and the pipeline was deployed locally. 
The integration of \ac{Ucloud} with \ac{HAAUKINS} and a pipeline remains uncertain. Out from the gathered 
material and information about \ac{Ucloud} it seems, that it is only feasible to utilize \ac{Ucloud} 
if \ac{DeiC} modifies its restrictions on internet access for spawned machines, 
allowing bidirectional traffic to and from machines on the \ac{Ucloud} infrastructure on other protocols than \ac{SSH}.

\section{HAAUKINS}
\label{sec:discussion-haaukins}
In this section, A presentment and discussion about the \ac{HAAUKINS} program. 
Also exploration of \ac{HAAUKINS} as a standalone program, separate from this 
specific project. 
The experiences and problems that has been encountered with \ac{HAAUKINS} during 
this project will be discussed in Section \ref{sec:discussion-pipeline}. 
Additionally, certain aspects of information will mentioned, that lack formal references, 
as they are either not covered in the \ac{HAAUKINS} documentation or were mentioned 
informally by other IT professionals to the author.\\
\ac{HAAUKINS} is a program developed by students at \ac{AAU},
and over the past year or two, it has been transitioning towards commercial use. 
During the time of writing this thesis, \ac{HAAUKINS} 2.0 has been under development and 
has been soft-released without any accompanying documentation. \\
Additionally, some details covered in this section might not be entirely accurate, 
as it has been impossible to contact the main developer of \ac{HAAUKINS}. 
There is no comprehensive documentation on how \ac{HAAUKINS} functions, 
so the only way to gather information is by reading the source code.
Through the work on the \ac{DDC} project, the author has gained some 
knowledge on \ac{HAAUKINS}.

\ac{HAAUKINS} is automated virtualization platform for security education. 
\ac{HAAUKINS} is constructed by three main components, Docker, VirtualBox, and \ac{Golang}.
\ac{HAAUKINS} utilizes \ac{Golang} to create the communication and orchestration between instances that 
it has running. The main reason of \ac{HAAUKINS} using \ac{Golang} as its programming language is its concurrency and 
parallelism mechanisms.\cite{haaukins}

\subsection{Docker}
\label{sec:haaukins-docker}
Docker is used for creating a closed network for the containers. Each challenge, called labs, that are 
spawned on \ac{HAAUKINS} has it own docker container. Using docker network in that
way creates a necessary security for the challenges, because users that are 
interacting with the challenges are not able to access other containers that are in use of 
other users.

To prevent other teams or players to interact with other teams or players labs,
\ac{HAAUKINS}
gives a team or player access to the specific docker network together with a kali linux machines, 
meaning that a team or player can interact with the labs, that he or she has spawned. 
In an earlier version of \ac{HAAUKINS} all labs were spawned at initialization. In \ac{HAAUKINS} 2.0
users are able to select which labs they want to spawn,
with a maximum of five labs. 
From personal experiences from the nationals \ac{DDC} and 
looking at the logs from the running \ac{HAAUKINS}, it is clear that the new version of \ac{HAAUKINS}
had a positive impact on the performance of \ac{HAAUKINS}.

\subsection{VirtualBox}
\label{sec:haaukins-virtualbox}
Virtualbox is used to manage \ac{VM}s. VirtualBox is one of the main components of \ac{HAAUKINS}.
Each user is given a kali linux machine, which contains all related tools to solve \ac{HAAUKINS} labs.
A user is given access to this kali linux machine \ac{GUAC}\cite{guacamole} interface, which is a web based interface
that allows the user to interact with the kali linux machine.

\subsection{\ac{Golang}}
\label{sec:haaukins-golang}
\ac{HAAUKINS} uses \ac{Golang} as it main programming language. As mentioned previously,
\ac{Golang} is used because of it concurrency and parallelism mechanisms. \ac{Golang} also 
provides Docker and virtualization libraries which makes the management of Docker containers and \ac{VM}s easier.

\subsection{HAAUKINS communication}
\label{sec:haaukins-communication}
There are number of cases where different types of protocols used to make communication reliable and consistent between components in \ac{HAAUKINS} platform. 
Mainly used protocols are\cite{haaukins-arch}
\begin{itemize}
    \item 
    \ac{HTTPS} is used to give the user access to CTFd. 
    CTFd is a web interface that tracks the overall progress of an event,
    and is also used to hand in flags.
    \item
    \ac{RDP}, is used for the communication to the \ac{VM}s. \ac{RDP} is used to communicate with the \ac{VM}s and the apache guacamole module.
    \item 
    \ac{gRPC} is used to communicate between the client and the daemon. \ac{gRPC} is a high performance, open-source and universal remote procedure call (RPC) framework.
\end{itemize}

\subsection{Creating a challenge in HAAUKINS}
\label{sec:haaukins-creating-challenge}
Through work on the \ac{DDC} project, knowledge was gained on how to create a challenge in \ac{HAAUKINS}. 
The process of creating a challenge on \ac{HAAUKINS} requires several steps.\cite{haaukins-challenge}

\paragraph{src folder \& solution folder}
The src folder contains the src code for the challenge that has been created. For example, 
if the created challenge is a web challenge using the flask framework, the src folder will contain 
the infrastructure for the flask framework, such as route, templates and static files.\\
The solution folder holds solution for the challenge if scripting or programming is required to solve the challenge.
\paragraph{Dockerfile}
The Dockerfile is placed in the root directory of the challenge. The Dockerfile is used to build the Docker image.
The Dockerfile is used by the pipeline in gitlab, for \ac{HAAUKINS} that file is called gitlab-ci.yml
\paragraph{gitlab-ci.yml}
\ac{HAAUKINS} is using pipeline to test integration for challenges into \ac{HAAUKINS}. This means 
that every time that a new branch is pushed to the main branch, gitlab will run the pipeline to test the integration of the challenge.
\paragraph{challenge-config}
The challenge-config folder contains the configuration files for a challenge on \ac{HAAUKINS}. 
Inside this folder, a file named challenge.yml is placed. The challenge.yml file includes various configuration details for the challenge, 
such as the name, category, difficulty, flag, readme, and solution. 
It also contains a complete description of the challenge and the text that users will see when they open the challenge.\\
For a haaukins challenge there is two types, \javaf{dynamic} and \javaf{static}.\\
\javaf{Dynamic challenge} is where 
an attacker has to interact with a remote host to get the flag. The remote host is packed inside a Docker container, and the 
attacker has access to the remote through Docker network as mentioned in Section \ref{sec:haaukins-docker}.\\
\javaf{Static challenge} consists of a downloadable artifact that the attacker has to analyze to get the flag. 
A challenge could either be a cryptography or a reverse engineering challenge, where the user gets files to analyze or reverse.\cite{haaukins-challenge}


\subsection{Local instance of HAAUKINS}
One of the aims of the project was to explore \ac{HAAUKINS} and determine the feasibility of deploying a local instance of 
\ac{HAAUKINS}. This instance was intended to be hosted on \ac{Ucloud} machines, with the pipeline and \ac{CTF}s running on that infrastructure.\\
A contact within the HAAUKINS project was available for assistance, but this individual primarily focused on developing 
events and \ac{CTF}s and lacked knowledge about local deployment of HAAUKINS.\\
As the exploration into deploying \ac{HAAUKINS} locally progressed, significant challenges emerged. 
The primary issue was the lack of documentation on local deployment. There was no clear information on how the daemon, 
client, and the CLI (hkn) worked together.

After a month of attempting to deploy \ac{HAAUKINS} locally, the effort was abandoned. 
The main reasons were the lack of documentation, insufficient information on the interaction between the different components of \ac{HAAUKINS}, 
and the inability to contact the main developer of \ac{HAAUKINS}.


\section{Pipeline}
\label{sec:discussion-pipeline}
Two pipelines using the same programs were developed during this project.
One pipeline with a local Drone instance and a remotely connected Gitea server.
The other is a totally localized deployable pipeline with a local Gitea server and a local Drone instance.
For structural purposes, the pipelines are named \javaf{local} and \javaf{remote}.

\subsection{Local pipeline}
\paragraph{Configuration at initialization}
The local pipeline resolved all the issues encountered with the remote pipeline, 
but it required creating several configuration files for initialization. To ensure the pipeline deployment was fully automated, 
all infrastructure components such as users, repositories, webhooks, and secrets had to be set up before the first login to Gitea and Drone.
Since the setup had to be completed before the actual Gitea and Drone servers were running, manual insertion into the database was necessary. 
This meant that for creating secrets, users, webhooks, and other configurations, 
it was essential to insert the data into the databases of both programs. 
This process often led to large SQLite databases queries with incorrect data entries and unpredictable program behavior.\\
Testing the extensive configuration required running the pipeline and all related programs in Docker containers. 
Each configuration and deployment cycle took four to six minutes. Given that the deployment process for the 
localized pipeline was executed around 300-500 times.
\paragraph{Resource consumption}
As the pipeline used Docker to spawn runners inside a Docker image, the resource consumption 
could potentially be high at a point. Drone gives a restriction in how many runners a pipeline or project 
can run a the same time, but there is no restriction in how many Docker containers a runners can spawn to be able to 
complete its execution. This could potentially lead to a high resource consumption on the host machine.
To prevent this a quota for the runners could be set, but this was not implemented in the pipeline.
Because the quota wasn't implemented into the runners was because a thorough testing of computational resources 
and how much a runner would consume was not done during extreme load. However, it is uncertain whether, if implemented, 
this features should be something done in \ac{HAAUKINS} or something done directly into the pipeline. Implementing
would also mean for local deployment that overusing resources would not be a problem.

\subsection{Remote pipeline}
\paragraph{OAuth2 issue}
The remote pipeline was constructed to address an issue encountered with OAuth2. 
The problem arose because the Drone server and the Gitea server were both running on the same machine using the same URL, 
localhost, differentiated only by port numbers. This caused OAuth2 to malfunction, as it could not distinguish between the two servers, 
resulting in redirects to the wrong server and failed authentication attempts.\\
Through extensive testing and debugging, the solution was found to separate the two servers and assign a 
different URL to the Gitea server. This was achieved by deploying the Gitea server on a remote machine while keeping the drone server 
on a local machine. This allowed OAuth2 to differentiate between the two servers, leading to a successful authentication process.\\
Later in the thesis process, it was discovered that this 
issue could also be resolved on the same machine by using different URLs for the two servers and placing them behind a reverse proxy.
\paragraph{Pipeline structure}
In the remote pipeline setup, the Drone server was localized while the Gitea server was remote. 
This arrangement made the connection and communication between the two servers challenging. 
Normally, both the Drone and Gitea servers would be exposed to the internet. 
However, automating the exposure of the Drone server to the internet was not resolved.\\
To facilitate communication between the two servers, an SSH connection from the Gitea server machine 
to the Drone server was used. This allowed the Gitea server to send \javaf{GET, POST, DELETE, and PUT} requests through 
the SSH connection to the Drone server. This approach was necessary because the Gitea server resided in a VM on \ac{Ucloud}, 
and, as mentioned in section \ref{sec:connection-ucloud}, the VMs on \ac{Ucloud} are only accessible via SSH.
After the \javaf{drone.sh} script was completed, users had to manually create a key pair for the Gitea server, 
enabling it to establish an SSH connection and forward traffic to the Drone server.

Having this structure for the pipeline was not optimal. It had many 
drawbacks and wasn't really an automated deployment. There was almost issues with either 
creating users, webhooks, push repositories to remote Gitea, pipeline not being able to sync with the Gitea server,
Gitea not being able to communicate with the Drone server and so on.
\paragraph{Ngrok issue}
For having the Gitea remotely accessible Ngrok was used to create a public \ac{URL} to the 
server and exposing it. Although Ngrok offers free usage with a single domain, a change in the Terms of Service (TOS) 
limited the amount of data available to free users per month. This limitation rendered the pipeline unusable once 
the data quota was exhausted, making the remote Gitea server inaccessible. 
Consequently, Ngrok was abandoned, and the pipeline was moved to a localized solution.



\section{Reflections and improvements towards CTF's}
\label{sec:discussion-future-work}
Although the project is complete, the discussion has highlighted several problems and challenges encountered during its development. 
This reflection has made it clear that there are still areas that need improvement and further work in the future. 

\subsection{Last CTF, Supply chain attack}
As mentioned at the end of section \ref{sec:ctf}, the last CTF in development 
was not finished due to time constrains.
This CTF is likely one of the more challenging ones, both in terms of implementation and for the user to solve. 
Completing this CTF would have been a valuable addition to the pipeline.
Although some of the development aspects of the creating CTF's was challenging,
there were two main challenges that were unable to be achieved.
\paragraph{Cron Job, for timed execution}
Whenever creating a \javaf{Cron Job} for Drone the software, whenever creating it through the API or the CLI,
adds the next execution time based on whenever the \javaf{Cron Job} was created.\\
However, since the \javaf{Cron Job} was created by inserting it directly into the database, 
the software was not present to do next execution time calculation and therefore the insertion of a wrong format in the database makes the 
UI break for the \javaf{Cron Job} when viewed on the Drone UI.
\paragraph{Implementation a exploit through the Docker image}
The proposed solution for the \ac{CTF} challenge was to modify the Docker image used by Ulduar in its pipeline. 
This Docker image could be updated through another repository Icecrown, 
which had a pipeline containing credentials capable of altering the Docker image in the registry.
However, during the development of the \ac{CTF}s, 
it became clear that the Drone pipeline removes any commands or environmental variables inserted into the Docker image it uses. 
Consequently, without knowing if there is an exploit of this nature for the Drone pipeline, implementing an exploit for this \ac{CTF} proved difficult.

Therefore in the end, it was not achieved to implement this \ac{CTF} in the pipeline. 
The source code for the \ac{CTF} is available in the repository, and it is possible to implement it in the future.

\subsection{\ac{CTF}s on the pipeline}
In the end of the thesis, it was possible for the pipeline to be tested by real users. 
As for most of the project, it was only the author that tested, the feedback from the users was valuable.
Here is some of the feedback given, but all the feedback is included in the appendix \ref{app:feedback}.
\begin{itemize}
    \item \textbf{Amirdrassil} 
    One of the users had a difficult time, figuring out where the flag was located, because there 
    was not any mention or use of any kind of security aspect in the pipeline. The user 
    expressed that it would be an idea to change the password for one of the services used in the pipeline, to 
    use the an environmental variable which was the flag.
    \item \textbf{Firelands}
    Having a file named "flag.py" makes every user just go to that file and grab the flag. 
    The flag should be hidden better. The feedback gives a general sense of the training challenges missed its mark, 
    because the user can just grab the flag without solving the challenge.
    \item Another mentioned that instead of just handing over the credentials to the user account 
    on Gitea and Drone, the user should have to figure out how to get the credentials.
\end{itemize}

The feedback indicates that the challenges in the pipeline does work, but when gamifying
it, is important that there is only one way to solve challenges. When users does \ac{CTF}, they 
will always choose the fastest option to exploit or solve the challenge.