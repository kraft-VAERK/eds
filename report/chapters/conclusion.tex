\chapter{Conclusion}
\label{sec:conclusion}

This thesis aimed to create a pipeline with 3-5 CTF challenges using \ac{HAAUKINS} and \ac{Ucloud}. 
The pipeline was built using Gitea, Drone, Docker, Nginx, Registry, and Docker Compose. 
It can be automatically deployed by any user on either Ubuntu or Kali Linux.

When incorporating \ac{HAAUKINS} and \ac{Ucloud} for the infrastructure, it became evident 
that using \ac{Ucloud} for the pipeline's computational power was not feasible. \ac{Ucloud}'s numerous restrictions, 
firewalls, and other security measures hindered control over the necessary resources, making its use impractical.

Regarding \ac{HAAUKINS}, it posed several challenges. My understanding of \ac{HAAUKINS} largely stemmed from the \ac{DDC} project. 
Initially, deploying the pipeline on \ac{HAAUKINS} was unachievable. Even after creating a localized solution, 
challenges remained due to network issues associated with \ac{HAAUKINS}, as discussed throughout the thesis. 
Additionally, after interacting with developers of HAAUKINS, it became clear that spawning HAAUKINS on \ac{Ucloud} was not possible. 
HAAUKINS requires access to the entire operating system to operate correctly, as discussed in section \ref{sec:discussion-haaukins}.

When comparing the two pipelines, it can be concluded that for CTF purposes, general use, 
and further development, a localized, self-deployed pipeline with Docker Compose is the best solution. 
The remote solution has some merit since it addressed the issue with OAuth2, but the problem with OAuth2 
was ultimately solved using a proxy and a \ac{TLD} in the localized pipeline.
However, the remote solution had too many flaws, 
including issues with SSH, SSH tunnels, webhooks, and the use of \ac{Ucloud} machines, which made it impossible to create any kind 
of smart communication between a local instance of Drone and a remote instance of Gitea.

With the use of a reverse proxy, it was possible for the \ac{CTF}s to be deployed locally. This meant that 
the \ac{CTF}s incorporated into the pipeline was able to be solved locally by anyone who deployed it.
The \ac{CTF}s in the pipeline are good, introducing Drone and providing general knowledge on how to interact with the pipeline. 

As the thesis was close to the end, the pipeline was tested by Brunnerne. That test proved 
that the pipeline was able to be deployed but other individuals than the author. 
From the feedback, it was clear that the pipeline was working as intended. 
The feedback mentioned that the challenges were nice, but they would like to see more technical challenges.

To summarize the thesis proposal, a pipeline incorporating CTF challenges was created. 
This pipeline is automated and can be deployed on the operating systems Ubuntu and Kali Linux. 
Regarding the use of \ac{HAAUKINS} and \ac{Ucloud}, we are currently planning to attempt deploying the pipeline 
on HAAUKINS after completing the thesis. However, due to numerous restrictions on machines in \ac{Ucloud}, 
utilizing \ac{Ucloud}'s computational power will not be feasible unless there are changes within its infrastructure. 
There is also potential for further development of CTF material for the pipeline based on 
feedback from Brunnerne, indicating room for improvement and additional content creation.