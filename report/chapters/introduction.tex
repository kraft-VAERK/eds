\chapter{Introduction}
For the last twenty years of software development, software developers have embraced the terminology of \ac{devops}.
\ac{devops} is a set of practices that combines software development (Dev) and IT operations (Ops). 
It aims to shorten the systems development life cycle and provide \ac{CI} and \ac{CD} with high software quality. 
\ac{devops} is complementary with Agile software development; several DevOps aspects came from Agile methodology. 

\ac{devops} started back in the early 21st century, and was first adapted by Amazon led by a man named Verner Vogels.
Vogels is considered the father of \ac{devops} by many software developers and IT professionals. From his perspective, 
he saw the need for a new way of working, where the development and operations teams would work together to
deliver software faster and more reliably. Instead of still using old ways of creating software 
like the waterfall model, where once a stage has been surpassed it's either impossible or very expensive to go back and change something.
\cite{devops}

Today, the new approach to software development, testing, and deployment has evolved into a structured process known as a pipeline. 
A pipeline consists of a series of automated steps that are executed whenever a developer merges code into a production branch or any 
branch connected to the pipeline, such as a test branch. While there are various ways to implement pipelines, the most common method involves 
using \ac{CI/CD} tools.

Nearly all software development companies today utilize some form of pipeline. These pipelines often contain sensitive information, 
including credentials, tokens, and other confidential data needed to deploy software to servers, cloud providers, or container registries. 
Due to the necessity of having access to these credentials and sensitive information, pipelines have become attractive targets for attackers. 
If an attacker gains access to a pipeline, they can potentially compromise the company's entire infrastructure.
Given this security risk, it is crucial to educate developers about potential security gaps in their pipelines and the bad practices they should avoid. 

In this thesis the focus is \ac{devops}. \ac{devops} introduces a new way of working using 
\ac{CI/CD}. Even though practices of \ac{CI/CD} are important, they are not the primary focus in this thesis. 
\ac{CI/CD} is a well-established practice, and numerous tools and platforms are available to facilitate its implementation.

The primary focus of this thesis will be, the creation of a pipeline and \ac{CTF} challenges that 
will help inform and educate developers about the security risks associated with pipelines. 
Currently, there a small amount of training material or \ac{CTF}'s to educate developers on securing their pipelines.
As of this reason, this project proposes the creation of a pipeline using open-source tools, 
designed with a set of challenges aimed towards improving pipeline security knowledge. 
Additionally, it is necessary to have a computational platform where this pipeline can be deployed, such as cloud computing.

HAAUKINS, an automated virtualization platform used by \ac{DDC} for hosting \ac{CTF} events, will be utilized to deploy and host these challenges. 
This platform will provide a practical environment for developers to learn and practice secure pipeline practices. 
Lastly, all this infrastructure will need some kind of computational power or cloud provider to run on. As 
\ac{SDU} is part of \ac{Ucloud}, a computational cloud provider, this thesis will explore the 
possibility of deploying the pipeline on \ac{Ucloud}.

At the end of the thesis, a pipeline locally deployable with multiple \ac{CTF}s on the pipeline was achieved.
Also the thesis concludes that the use of \ac{HAAUKINS} and \ac{Ucloud} was not possible.
Through GameSS\cite{gamess} it was possible to have
the pipeline tested by a group called Brunnerne. Brunnerne is a local group of people from Odense and \ac{SDU}, that are
enthusiastic about \ac{CTF}. This testing provided valuable feedback on the pipeline as the group were appreciated 
by the existing new form of \ac{CTF} challenges but expressed a desire to see more challenges with a more technical focus.

The structure of the thesis is as follows:
\begin{itemize}
    \item Chapter 1 \javaf{State of the art}, An explanation of tools and other pipeline technologies that are out there.
    \item Chapter 2 \javaf{Techninal Implementation} - The thesis has two implementations. One pipeline that was created with a remote server, and 
    one that is totally localized.
    \item Chapter 3 \javaf{HAAUKINS and Ucloud} - In this chapter the two platforms are introduced and explained.
    \item Chapter 4 \javaf{Discussion} - In the discussion there will be a focus on the challenge faced during the projects.
    Also there will be a reflection and a future work discussion.
    \item Chapter 5 \javaf{Conclusion} - An overall conclusion that reflects on the goals of the project.
\end{itemize}
The source code for this project can be found online at \url{https://git.imada.sdu.dk/mojak18/Empowering_DevOps_Security}