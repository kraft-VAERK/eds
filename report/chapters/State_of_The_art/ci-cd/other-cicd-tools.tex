\section{CI/CD CTF Pipeline}
\label{sec:ci-cd-ctf-pipeline}
As this project focuses on the development of a pipeline for the creation and deployment of \ac{CTF} challenges, 
it is essential to understand the available pipeline platforms, that are already out there. 
Even though there may not be an existing pipeline exactly 
like the one being developed, it is still important to understand what other tools can and cannot do. 
In this section, I will briefly some of the pipelines that I consider noteworthy. 
These pipelines have some merits in terms of what I believe a pipeline should be able to do and what it should not do.


\paragraph{CICD-Goat}
\label{par:cicd-goat}
The CICD-Goat\cite{cicd-goat} is one of the other \ac{CTF} pipelines that is out there.
The pipeline is created based on the OWASP Top 10 CI/CD security risk\cite{owasp-cicd-top-10} list.
The pipeline is aimed toward developers, security professionals, and others who are interested in understanding the \ac{CI/CD} pipeline.
The pipeline is constructed with technology of Gitea, Jenkins, Gitlab and Gitlab runners.
The pipeline infrastructure is build inside Docker containers and used localhost to export the website.
It used a local instance of CTFd to host the challenges and report the flag.

The CICD-goat project has a lot of merit in the sense that they focus on real proven scenarios that 
is happening to pipelines in productions settings, anyhow the project uses an old technology in Jenkins. Jenkins 
is a great tool, but in the sense of a modern pipeline and pipeline structure, Jenkins is outdated.

\paragraph{Poursoft}
\label{par:poursoft}
Poursoft is the recent create pipeline for the national \ac{DDC}. It was created by Oliver Nordestgaard
and was used in the national \ac{DDC} 2024. The pipeline utilize many of the same tools as the pipeline for this project.
The pipeline is not open-source and can't be referenced in this project. Although through my work on \ac{DDC} 2024,
I had access to the repository and could see how the pipeline was constructed. The pipeline was constructed with the following tools:
\begin{itemize}
    \item \javaf{Gitea} as the git repository.
    \item \javaf{DroneCI} for the pipeline
    \item \javaf{Ngnix} for a websever and reverse proxy
    \item \javaf{Haaukins} for the deployment of the challenges
\end{itemize}

\paragraph{Tryhackme, Hackthebox, and other commercial projects}
\label{par:tryhackme-hackthebox}
TryHackMe and Hack The Box are both commercial platforms and are not open-source. They require a subscription fee for access.
On TryHackMe\cite{tryhackme}, it is possible to view available challenges without logging into the website. 
TryHackMe offers numerous challenges relevant to pipelines. While not directly focused on pipelines, 
these challenges involve tools commonly used in pipelines, such as Docker, Kubernetes, GitLab, DevSecOps practices, and more.\\
Hack The Box\cite{hackthebox} primarily focuses on the exploitation of machines. After exploring their website, it appears they have a few 
DevOps-related challenges. However, 
the specific focus of these challenges is not clearly indicated, making it difficult to assess their relevance to DevOps Capture The Flag (CTF) events.

