\subsection{Drone CI/CD}
\label{sec:drone}
Drone\cite{droneio} is a \ac{CI/CD} tool that is used to automate the process of building, testing and deploying software.
Drone is a container-native \ac{CI/CD} platform built on Docker, written in Go. It uses a simple \ac{YAML} configuration file
to define the pipeline. 
Drone is open-source. Everything about the DroneAPI is open-source and 
available on their website.
As Gitea has CI/CD support, what that means is that 
it is possible to have Drone as an external pipeline software work together with Gitea.
\begin{figure}
    \begin{minted}[
        gobble=4,
        frame=single,
        linenos
      ]{yaml}
        kind: pipeline
        type: docker
        name: default
            
        steps:
        - name: backend
          image: golang
          commands:
          - go build
          - go test
            
        - name: frontend
          image: node
          commands:
          - npm install
          - npm run test
      \end{minted}
      \caption{Drone pipeline example}
      \label{fig:drone_pipeline}
\end{figure}
An example of a simple 
pipeline that runs a \javaf{go build} and \javaf{go test}. Then after it has run the first step, 
it will run the second step which is a \javaf{npm install} and \javaf{npm run test}.\\
The drone pipeline is defined from the \javaf{.drone.yml} file. 
This file is located in the root of the repository.
The pipeline is defined in the file, and the pipeline is then executed by the drone runners.
The pipeline uses keywords to determine it actions and then executes them based on 
the input in the pipeline and the data in the repository.
In the DroneCI documentation\cite{droneio}, there is a list of 
keywords that are available for use in the pipeline. The main keywords applied in the project are\cite{droneio}:

\begin{itemize}
  \item \textbf{kind} 
  The kind setting is to define to the runner that it is a pipeline. The setting is default and the only setting for this 
  keyword.
  \item \textbf{type}
  The type declare for the backend (drone server), what the pipeline infrastructure is. Drone provides several types of types for pipelines such as 
  k8s\cite{kubenetes}, Docker, exec, ssh and more. The most common is Docker.
  \item \textbf{name}
  The name of the pipeline used to identify the pipeline in the drone server.
  \item \textbf{steps}
  Steps is the main part of the pipeline. In this part commands are defined. 
  The steps are executed in order from top to bottom.
  During a steps sequence you can have multiples step, inside a step you have new keywords, the most commonly used are:
    \begin{itemize}
      \item \textbf{name}
      Name of the step you're executing.
      \item \textbf{image}
      This is specific for the Docker pipeline. This is the image that the step will run in. 
      The drone server uses Docker hub to collect the images if nothing else is specified.
      \item \textbf{commands}
      The commands that the step will run. This is the same as running the commands in a terminal.
    \end{itemize}
  \item{\textbf{environment}}
  The environment keyword is used to define environment variables that are available to all steps in the pipeline.
  It is attached to a step in the pipeline and it is used to make secrets and other variables available to the steps.
\end{itemize}

These are examples of the keywords that are used by the drone pipeline file, to execute the pipeline.
In this project these are the main ones that are used.
\subsection{Drone runners}
To enable the Drone pipeline's functionality, runners are necessary to execute commands triggered by the pipeline. 
The Drone server acts as the intermediary between Gitea and these runners. These runners are configured as Docker containers. 
Upon startup, the drone runners attempt to connect to the server. If the ping is successful, the runner gains the capability to execute the pipeline.

Configuring drone runners is straightforward since most of the setup resides on the Drone server and within the YAML configuration file. 
The runner itself is merely a container initialized with the drone-runner image. This image is provided by the Drone pipeline configuration file, as depicted in figure \ref{fig:drone_pipeline}. 
Specific configurations for a drone-runner will be detailed in section \ref{sec:pipeline-local}.