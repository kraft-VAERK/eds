\subsection{Gitea - Git Server}
\label{sec:gitea}
\subsubsection{Gitea}

Gitea\cite{giteadocs} is a self-hosted \ac{Git}\cite{git} service. Similar to GitHub\cite{GitHub}. 
It allows the user to customize and run 
their own Git server. Gitea has a lot of great features such as: code hosting, code review, issue tracking, and wiki pages.\\
Gitea offers the possibility of authenticating a \ac{CI/CD} tool to the server. This means that the user
can have a external pipeline software such as Jenkins\cite{jenkinsio}, GitLab CI\cite{gitlab} or Drone\cite{droneio} to run tests on the code before it is
merged into the master branch.
Gitea is open-source, meaning that users will be able to customize it completely to their needs.

\subsubsection{Gitea for this project}
Firstly, Gitea is open-source and customization is key since it gives the possibilities to change things based on vulnerabilities and 
make Gitea acts as the CTF needs.
Secondly, it has \ac{CI/CD} support. Meaning that \ac{CI/CD} tools such as Drone can be integrated with Gitea.
Thirdly, it has code review. Code review is important since 
since the pipeline can be configured in so many ways, and the runners for the pipeline will most likely execute code, 
with either all privileges or at least with some more than a developer has. This could mean that there might be some way 
for a malicious user to exploit the pipeline and gain access to runner based on just asking for code reviews.
\subsubsection{Gitea static installation}
For this project, the aim is to offer a streamlined 
installation process for Gitea.
Based on the fact that Gitea can be customized to the needs of this project,
a static installation which depend on only itself is the best approach.
This involves providing a static installation 
tailored specifically for deployment via Docker.
That includes, initialization, basic configuration, minimal infrastructure setup, users, Oauth2, repositories, and more.\\
This ensure that once the Gitea is deployed and running, that a user 
do not have to do anymore configuration and can start using the Gitea server right away, for it the desired purpose.\\
All this needs to be encapsulated inside the Docker image, that essential will spawn the Gitea server.
Therefore the following is important for the Gitea server: 

\begin{itemize}
    \item \textbf{Version} The version for gitea is important to predefined. 
    This is because Gitea is an open-source project and can be subjective to changes at any time. For this project,
    it is pivotal to ensure that the end result is the same as when the project was started.
    The version used is $1.21.6$. This is specified in the \javaf{Dockerfile.gitea}.
    \item \textbf{Configuration} The configuration of Gitea needs to be static.
    When Gitea is installed,
    it will prompt the installer with a initialization configuration, see figure \ref{fig:gitea_app_ini_remote} for example,
    which is then used to configure Gitea. In this configuration there is the base information for Gitea to run properly.
    As the deployment of Gitea must be automated, a predefined configuration of the \javaf{app.ini}
    file is injected into the Docker image, which then will be used to configure Gitea.
    can be found the source code of the project under \javaf{src/Gitea/app.ini}.
    The configuration of Gitea also includes a minimal infrastructure setup. What this means is 
    that the database can be accessed before the Gitea server is started.

    \item \textbf{Repositories} Gitea is the code base of the CTF's repositories. 
    This means that the repositories needs to be created at initialization, so that when the user start the image via docker-compose,
    all the repositories are present.
    The repositories are predefined in \javaf{src/Gitea/repositories}.
    The repositories are pushed to the Gitea server database via a python3 script called \javaf{giteacasc}\cite{cicd-goat}.
\end{itemize}
\begin{figure}[h]
    \begin{center}
        \includegraphics[scale=.5]{images/Gitea-appin.JPG}
        \caption{Gitea app.ini}
        \label{fig:gitea_app_ini_remote}
    \end{center}
\end{figure}
There is a lot of other configuration needed specifically for Gitea, but these will be discussed in section \ref{sec:pipeline-local}.
\subsubsection{Gitea remote}
In this project, a local instance of Gitea was made remotely accessible, effectively turning it into a public Gitea server. 
This was achieved using Ngrok\cite{ngrok}, an ingress service that exposes the local Gitea server to the internet. 
Ngrok will be described in section \ref{sec:ngrok}. The local and remote Gitea server had very different configurations; the remote 
Gitea need to be configured whenever a new user arrived, whereas the local Gitea instance needed to be fully configured at initialization. 
Recent updates to Gitea have restricted the use of admin calls, necessitating the use of an earlier version for remote configuration. 
The remote version of Gitea was $1.16.5$, allowing for more manipulation of Gitea data, through the \ac{API},
compared to the localized version, which used the later version $1.21.6$.
