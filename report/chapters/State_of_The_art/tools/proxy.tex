\subsection{Nginx proxy}
\label{sec:nginx}
Nginx is a reverse proxy, serving as the intermediary between the client and the internal network where the application runs. 
It manages the DNS records requested and directs the traffic to the corresponding IP address. 
Consequently, the client is unaware of the application's IP address, knowing only the gateway. This gateway is a docker network gateway.
The client sends a request to the application via HTTPS. 
The proxy forwards this request to the application, which then sends the response back to the proxy. 
This setup adds an extra layer of abstraction, helping to protect the internal network from the outside world.
It also gives the possible to add more infrastructure to the application, such as load balancing, caching, certificates handling and more.\\

\paragraph{Passing a request to a proxied server}
Whenever nginx receives a request, it tries to find the server that the request should be forwarded to.
The server is specified in the \javaf{proxy_pass} directive. The \javaf{proxy_pass} directive is used to specify the server that the request should be forwarded to.\\
Simply what the \javaf{proxy_pass} specification does, is the declaration of the server that the request should be forwarded to.
That can either be a \ac{DNS} record or an \ac{IP} address.\cite{nginx-proxy-3}

\paragraph{Configuring buffers}By default NGINX buffers responses from upstream servers.
A response is stored in the internal buffers and is not sent to
the client until the whole response is received. Buffering helps to optimize performance with slow clients, 
which can waste upstream servers time if the response is passed from NGINX to the client synchronously. 
However, when buffering is enabled, NGINX allows the upstream servers to 
process responses quickly, while NGINX stores the responses for as much time as the clients need to download them.\cite{nginx-proxy-3}
The specification of these buffers are noted in the configuration as seen in figure \ref{fig:buffer-config}.
\begin{figure}
    \begin{center}
        \javaf{proxy_buffers $size$ $number$;} \\
        \javaf{proxy_buffer_size $size$;}.
    \end{center}
    \caption{Configuring buffers}
    \label{fig:buffer-config}
\end{figure}
The specific configuration for the this project will be covered in section \ref{sec:pipeline-local-nginx}.

\subsubsection{Docker gateway}
\label{sec:gateway}
For the NGINX reverse proxy to function, it needs a gateway to attach itself to. Here the Docker network is used as the gateway.
The Docker network is a virtual network that is created by Docker. The network is used to connect containers together.
The network is created by the Docker daemon when the first container is created. The network is created with the name \javaf{bridge}.
The bridge network is the default network that is created by Docker.
The bridge network is a \ac{NAT} network that is used to connect the containers to the host.