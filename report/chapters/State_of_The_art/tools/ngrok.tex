\subsection{Ngrok}
\label{sec:ngrok}
Ngrok was used is this product to expose the local Gitea server to the internet. This was done to allow remote access to the server.
Ngrok is a multiplatform tunnelling, reverse proxy software that establishes secure tunnels from a public endpoint to a local running network service.
Ngrok acts as a unified ingress platform combining all the components to deliver traffic from local service to the internet. 
Ngrok consolidates together the reverse proxy, load balancer, \ac{API} gateway, firewall, delivery network, DDoS protection and more.\cite{ngrok}

\subsubsection{How does Ngrok work?}
Ngrok is a reverse proxy but there is a little more to it than just that. Ngrok 
works on a global network called the \javaf{Ngrok edge}. At the edge, traffic is accepted from client from the internet 
and then forwarded to the local service.
Unlike traditional proxies Ngrok does not transmit traffic to an upstream application by forwarding IP address. Instead, 
Ngrok is a small piece of software that you run along side a local application. This software will connect to 
the Ngrok edge and will keep the connection open. When a client connects to the Ngrok edge, the edge will forward the
traffic to the local application.
There are four ways of running the client\cite{ngrok-works}.

\begin{itemize}
    \item As a service: Run as a small side process called the Ngrok agent as a
    background \ac{OS} service.
    \item As an interactive \ac{CLI}: Run as the Ngrok agent
    interactively from the command line while developing and testing.
    \item As an \ac{SDK}
    embedded in your app: Included as a small Agent \ac{SDK} library directly into an
    application software that returns a socket-like object. 
    \item As a Kubernetes
    Controller: Run the Ingress Controller in a Kubernetes environment.
\end{itemize}

In this project, the Ngrok agent was run as a service. This was done to ensure that the connection to the Ngrok edge was at all times.