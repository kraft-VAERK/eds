\subsection{Registry}
\label{sec:registry}
The Drone pipeline when using Docker runner needs a registry to pull and push images to.
The default registry that is used is the Docker hub. Because the future work of the project includes
having the project running on haaukins (Section \ref{sec:haaukins}) which does not have internet when running.
Here in comes a local registry. A local registry will resides as a mirror of Docker Hub, when 
other container which has docker installed is made aware of the local registry, it will pull images from the local registry.
If no internet connection is available, the Drone runners will not be able to pull images from the docker hub. 
How this is solved is either by providing all container with the address of the mirror registry of Docker hub or 
using specific \ac{URL} for the desired images in the pipeline.\\
\paragraph{What is a registry?}
A registry is a storage and content delivery system, holding named Docker images, available in different tagged versions.
The registry is a stateless, highly scalable server side application that stores and let you distribute Docker images.
The registry is open-source, under the Apache 2.0 license. Some of the features of the registry are:

\begin{itemize}
    \item The ability to store images in a central location.
    \item The ability to control access to the images.
    \item The ability to integrate image storage and distribution into the \ac{CI/CD} pipeline.
\end{itemize}

\newpage