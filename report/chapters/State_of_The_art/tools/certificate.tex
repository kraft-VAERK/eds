\subsection{Certificate}
\label{sec:certificate}
Certificates are used for all websites that run \ac{HTTPS}. 
The certificate is used  to let visitors know that the website is secure.
The certificate has associated cryptographic keys that are used to encrypt the data being transferred between the client and the server.
Futhermore, the certificate might be signed by a \ac{CA} which is a trusted third party.
When a client connects to a website that uses \ac{HTTPS}, the client will check the certificate
to see if it is signed by a trusted \ac{CA}. If the certificate is signed by a trusted \ac{CA}, the client
will trust the website. If the certificate is not signed by a trusted \ac{CA}, the client will get a warning
that the website is secured by a self signed certificate.\\
There are various methods to create a certificate, and no single approach is definitive. 
For this project, a self-signed certificate was created to enable HTTPS connections between all services \cite{self-signed-cert}.


\paragraph{Self signed certificate}
When creating a self signed certificate, access to the server or the proxy that is going to handle the certificate is necessary.
The certificate is created using the openssl command first one will have to create a private key:
\begin{figure}[h]
    \begin{center}
        \javaf{openssl genrsa -out eds.key 2048}
    \end{center}
    \caption{Creating a private key}
    \label{fig:private-key}
\end{figure}
The private key created in figure \ref{fig:private-key} is a RSA key that is 2048 bits long. 
Normally, the private key would be encrypted with a password, but for this project, 
the private key will not be encrypted to avoid the need to enter a password during deployment.
Next, a Certificate Signing Request (CSR) will be created. 
A CSR is an encoded file containing information about the organization requesting the certificate \cite{csr}. 
The CSR is sent to a Certificate Authority (CA) for signing. In this case, the certificate will be self-signed.
\begin{figure}[h]
    \begin{center}
        \javaf{openssl req -key eds.key -new -out eds.csr}
    \end{center}
    \caption{Creating a CSR}
    \label{fig:csr-creation}
\end{figure}
When creating a \ac{CSR}, openssl ask for information to be filled. This information is about the creator of the CSR.
The information that is asked for is:
\begin{itemize}
    \item Country Name (2 letter code) [AU]: 
    \item State or Province Name (full name): 
    \item Locality Name (eg, city): 
    \item Organization Name (eg, company): 
    \item Organizational Unit Name (eg, section): 
    \item Common Name (e.g. server FQDN or YOUR name): 
    \item Email Address:
    \item A challenge password:
    \item An optional company name:
\end{itemize}
\ac{CN} is the domain name which the certificate is going to be used for.
After the private key and \ac{CSR} is created, we can create the certificate. We created the certificate using \javaf{x509} format.\\
\javaf{x509}\cite{x509} is standardized format for public key certificates. What \javaf{x509} certificate provides is the ability
to both create a self signed certificate, but also allow our application to be accessed through \ac{HTTPS}.\\
\begin{figure}[h]
    \begin{center}
    \javaf{openssl x509 \ }\\
    \javaf{-signkey eds.key \ }\\
    \javaf{-in eds.csr \ } \\
    \javaf{-req -days 365 -out eds.crt}
\end{center}
    \caption{Creating a certificate}
    \label{fig:cert-creation}
\end{figure}
The certificate, created by the command seen in figure \ref{fig:cert-creation}, is designed for a single page. 
To support multiple websites under various domains within the same subdomain, 
a certificate endorsed by a Certificate Authority (CA) is required.\\
Having the created certificate signed by 
a CA allows the association of the desired subdomain, enabling an unlimited number of subdomains. 
The initial step involves generating a new private key and self-signing 
an CA certificate, which will be used to endorse the previously created certificate.
Now that a CA certificate called edsCA is created by the command seen in figure \ref{fig:ca-cert}.
edsCA can be used to sign the certificate that was created in figure \ref{fig:cert-creation}.\\
\begin{figure}[h]
    \begin{center}
        \javaf{openssl req -x509 -sha256 -days 1825 \ } \\
        \javaf{-newkey rsa:2048 -keyout edsCA.key -out edsCA.crt}
    \end{center}
    \caption{Creating a CA certificate}
    \label{fig:ca-cert}
\end{figure}
A configuration file, named \javaf{eds.ext}, will be used to add additional information to the certificate. 
The \javaf{eds.ext} file contains the following information:
\begin{figure}[h]
    \begin{center}
        \javaf{basicConstraints=CA:FALSE} \\
        \javaf{extendedKeyUsage=serverAuth} \\
        \javaf{subjectAltName = *.devops.eds}
    \end{center}
    \caption{Data for configuration file for the CA certificate}
    \label{fig:ca-cert-ext}
\end{figure}
\javaf{extendedKeyUsage} is serverAuth because we're going to use ssl/TLS authentication a server.
\javaf{basicConstraints} is set to false because the certificate is not a valid  CA certificate.\\
\javaf{subjectAltName} is the domain name that the certificate is going to be used for.\\\cite{openssl-ext}
\begin{figure}[h]
    \begin{center}
        \javaf{openssl x509 -req -CA edsCA.crt \ } \\
        \javaf{-CAkey edsCA.key -in eds.csr \ } \\
        \javaf{-out eds.crt -days 365 \ } \\
        \javaf{-CAcreateserial -extfile eds.ext}
    \end{center}
    \caption{Signing eds.crt with edsCA.crt}
    \label{fig:sign-cert}
\end{figure}

The command from figure \ref{fig:sign-cert} takes a CSR (eds.csr), signs it using the CA certificate 
(edsCA.crt) and its private key (edsCA.key), and generates a signed certificate 
(eds.crt) valid for 365 days, including any specified extensions from the eds.ext file.\\
This certificate has been signed with the CA certificate, 
meaning that if it is placed in \javaf{/etc/ssl/certs/ca-certificates.crt} on Linux, the system will trust the certificate.