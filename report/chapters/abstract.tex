\begin{abstract}
    I denne afhandling er der blevet undersøgt og udviklet en pipeline som kan bruges til 
    at generere, træne og vise \ac{CTF}'s i et \ac{devops} miljø. Pipelinen 
    er opbygget af 5 forskelle programmer og kan køres lokalt, 
    på computere der har operativ system kali eller ubuntu linux.
    
    Det overordnede mål for denne afhandling var først og fremmest at undersøge, mulighed for at bruge cloud computing 
    til pipeline-infrastrukturen. Derudover at se om et automatiseret virtualiseringsprogram kan bruges til at 
    beværte pipelinen, hvilket på sigt vil gøre det muligt for både brugere og udviklere at håndtere pipelinen og dens ressourcer. 
    Derudover. har det været i fokus at gøre en pipeline tilgængelig for brugeren, sammen med 
    undervisningsmateriale i form af CTF'er, som kan hjælpe brugeren med at forstå pipelinen og samtidig lære om sikkerhed i pipelines.
    
    I denne afhandling er produktet, der er blevet fremstillet, en pipeline.
    Pipelinen kan køres lokalt og opstartes ved bruge at et skript. 
    Fuldføres skriptet uden fejl, vil brugeren blive præsenteret med tre links.
    Disse tre links er til Gitea, Drone og Registry. 
    Disse er de tre programmer som brugeren kan interagere med for at bruge pipelinen,
    og i disse tre programmer er der 4 CTF'er som brugeren kan løse.
    \\
    Udover fremstilling af pipelinen, blevet det også en mulighed at undersøge om "cloud computing"
    kunne anvendes.
    Dette blev undersøgt ved at bruge en platform kaldet Ucloud.Ucloud's maskiner 
    er tilgængelige ved brugen at en SSH forbindelse. På grund af denne begrænsning på 
    Ucloud maskiner, var det ikke muligt at bruge Ucloud som en ressource til at køre infrastrukturen for pipelinen.
    \javaf{HAAUKINS} skulle bruges til at hoste pipelinen og gøre det let for brugerne 
    at intergerer med det udviklede materiale.
    \javaf{HAAUKINS} programmet blev undersøgt og 
    det viste sig, at med den minimale dokumentation og manglede support fra udviklerne, 
    at det ikke var muligt at få haaukins til at køre lokal, og derved ikke have nogen som 
    kontrol over programmet som afhandlingen havde brug for.\\
    I sidste ende kan det konkluderes, at den udviklede pipeline kan køres og startes totalt lokalt, og at brugeren
    kan anvende pipelinen til at løse de tilgængelige \ac{CTF}'er.
\end{abstract}
\begin{abstract}
    This thesis set to investigated and develope a pipeline that can be used to generate, train, and display Capture The Flags (CTFs) in a DevOps environment. 
    The pipeline is composed of 5 different programs and can be run locally on computers with Kali or Ubuntu Linux 
    operating systems.

    The aim goal of this thesis was to investigate if it would be possible 
    to use cloud computing for the pipeline's infrastructure. Secondly, to investigate whether it is possible to use an automated 
    virtualization program to host the pipeline, which effectively allows both the developer to manage the pipeline and its resources,
    and user to interact with it easily. 
    Last but not least, to investigate and develop a pipeline that can be locally accessible to the user, along with developing educational material 
    in the form of CTFs that could be used to enable the user to understand the pipeline while also learning about security in pipelines.

    The result of this thesis is a pipeline where the user can run a simple script. 
    This script will automatically check and verify if all necessary programs are installed. 
    If the script completes without errors, the user will be presented with three links: Gitea, Drone, and Registry. 

    Besides generating a pipeline, the possiblity of using cloud computing was investigated.
    This was explored by using a program called Ucloud.
    Ucloud's machines are only accessible through a SSH connection, 
    it was not possible to use Ucloud as computing power to run the 
    infrastructure for the pipeline, due to this limitation.

    The Haaukins program was intended to host the pipeline and make it easy for users to interact with the pipeline. 
    The Haaukins program was investigated and it turned out that due to minimal documentation and lack of support from the developers,
    it was not possible to run Haaukins locally, and therefore, was the aim of having the program run not achievable.

    Ultimately, it can be concluded that the pipeline can be run locally and that the user can use the pipeline to solve
    the available \ac{CTF}s that are present on the pipeline. 
    It can also be concluded that it was not possible to use Ucloud or Haaukins to host the pipeline.
    
\end{abstract}


\newpage