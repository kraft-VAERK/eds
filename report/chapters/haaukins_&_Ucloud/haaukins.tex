In this section, A presentment and discussion about the \ac{HAAUKINS} program. 
Also exploration of \ac{HAAUKINS} as a standalone program, separate from this 
specific project. 
The experiences and problems that has been encountered with \ac{HAAUKINS} during 
this project will be discussed in Section \ref{sec:discussion-pipeline}. 
Additionally, certain aspects of information will mentioned, that lack formal references, 
as they are either not covered in the \ac{HAAUKINS} documentation or were mentioned 
informally by other IT professionals to the author.\\
\ac{HAAUKINS} is a program developed by students at \ac{AAU},
and over the past year or two, it has been transitioning towards commercial use. 
During the time of writing this thesis, \ac{HAAUKINS} 2.0 has been under development and 
has been soft-released without any accompanying documentation. \\
Additionally, some details covered in this section might not be entirely accurate, 
as it has been impossible to contact the main developer of \ac{HAAUKINS}. 
There is no comprehensive documentation on how \ac{HAAUKINS} functions, 
so the only way to gather information is by reading the source code.
Through the work on the \ac{DDC} project, the author has gained some 
knowledge on \ac{HAAUKINS}.

\ac{HAAUKINS} is automated virtualization platform for security education. 
\ac{HAAUKINS} is constructed by three main components, Docker, VirtualBox, and \ac{Golang}.
\ac{HAAUKINS} utilizes \ac{Golang} to create the communication and orchestration between instances that 
it has running. The main reason of \ac{HAAUKINS} using \ac{Golang} as its programming language is its concurrency and 
parallelism mechanisms.\cite{haaukins}

\subsection{Docker}
\label{sec:haaukins-docker}
Docker is used for creating a closed network for the containers. Each challenge, called labs, that are 
spawned on \ac{HAAUKINS} has it own docker container. Using docker network in that
way creates a necessary security for the challenges, because users that are 
interacting with the challenges are not able to access other containers that are in use of 
other users.

To prevent other teams or players to interact with other teams or players labs,
\ac{HAAUKINS}
gives a team or player access to the specific docker network together with a kali linux machines, 
meaning that a team or player can interact with the labs, that he or she has spawned. 
In an earlier version of \ac{HAAUKINS} all labs were spawned at initialization. In \ac{HAAUKINS} 2.0
users are able to select which labs they want to spawn,
with a maximum of five labs. 
From personal experiences from the nationals \ac{DDC} and 
looking at the logs from the running \ac{HAAUKINS}, it is clear that the new version of \ac{HAAUKINS}
had a positive impact on the performance of \ac{HAAUKINS}.

\subsection{VirtualBox}
\label{sec:haaukins-virtualbox}
Virtualbox is used to manage \ac{VM}s. VirtualBox is one of the main components of \ac{HAAUKINS}.
Each user is given a kali linux machine, which contains all related tools to solve \ac{HAAUKINS} labs.
A user is given access to this kali linux machine \ac{GUAC}\cite{guacamole} interface, which is a web based interface
that allows the user to interact with the kali linux machine.

\subsection{\ac{Golang}}
\label{sec:haaukins-golang}
\ac{HAAUKINS} uses \ac{Golang} as it main programming language. As mentioned previously,
\ac{Golang} is used because of it concurrency and parallelism mechanisms. \ac{Golang} also 
provides Docker and virtualization libraries which makes the management of Docker containers and \ac{VM}s easier.

\subsection{HAAUKINS communication}
\label{sec:haaukins-communication}
There are number of cases where different types of protocols used to make communication reliable and consistent between components in \ac{HAAUKINS} platform. 
Mainly used protocols are\cite{haaukins-arch}
\begin{itemize}
    \item 
    \ac{HTTPS} is used to give the user access to CTFd. 
    CTFd is a web interface that tracks the overall progress of an event,
    and is also used to hand in flags.
    \item
    \ac{RDP}, is used for the communication to the \ac{VM}s. \ac{RDP} is used to communicate with the \ac{VM}s and the apache guacamole module.
    \item 
    \ac{gRPC} is used to communicate between the client and the daemon. \ac{gRPC} is a high performance, open-source and universal remote procedure call (RPC) framework.
\end{itemize}

\subsection{Creating a challenge in HAAUKINS}
\label{sec:haaukins-creating-challenge}
Through work on the \ac{DDC} project, knowledge was gained on how to create a challenge in \ac{HAAUKINS}. 
The process of creating a challenge on \ac{HAAUKINS} requires several steps.\cite{haaukins-challenge}

\paragraph{src folder \& solution folder}
The src folder contains the src code for the challenge that has been created. For example, 
if the created challenge is a web challenge using the flask framework, the src folder will contain 
the infrastructure for the flask framework, such as route, templates and static files.\\
The solution folder holds solution for the challenge if scripting or programming is required to solve the challenge.
\paragraph{Dockerfile}
The Dockerfile is placed in the root directory of the challenge. The Dockerfile is used to build the Docker image.
The Dockerfile is used by the pipeline in gitlab, for \ac{HAAUKINS} that file is called gitlab-ci.yml
\paragraph{gitlab-ci.yml}
\ac{HAAUKINS} is using pipeline to test integration for challenges into \ac{HAAUKINS}. This means 
that every time that a new branch is pushed to the main branch, gitlab will run the pipeline to test the integration of the challenge.
\paragraph{challenge-config}
The challenge-config folder contains the configuration files for a challenge on \ac{HAAUKINS}. 
Inside this folder, a file named challenge.yml is placed. The challenge.yml file includes various configuration details for the challenge, 
such as the name, category, difficulty, flag, readme, and solution. 
It also contains a complete description of the challenge and the text that users will see when they open the challenge.\\
For a haaukins challenge there is two types, \javaf{dynamic} and \javaf{static}.\\
\javaf{Dynamic challenge} is where 
an attacker has to interact with a remote host to get the flag. The remote host is packed inside a Docker container, and the 
attacker has access to the remote through Docker network as mentioned in Section \ref{sec:haaukins-docker}.\\
\javaf{Static challenge} consists of a downloadable artifact that the attacker has to analyze to get the flag. 
A challenge could either be a cryptography or a reverse engineering challenge, where the user gets files to analyze or reverse.\cite{haaukins-challenge}


\subsection{Local instance of HAAUKINS}
One of the aims of the project was to explore \ac{HAAUKINS} and determine the feasibility of deploying a local instance of 
\ac{HAAUKINS}. This instance was intended to be hosted on \ac{Ucloud} machines, with the pipeline and \ac{CTF}s running on that infrastructure.\\
A contact within the HAAUKINS project was available for assistance, but this individual primarily focused on developing 
events and \ac{CTF}s and lacked knowledge about local deployment of HAAUKINS.\\
As the exploration into deploying \ac{HAAUKINS} locally progressed, significant challenges emerged. 
The primary issue was the lack of documentation on local deployment. There was no clear information on how the daemon, 
client, and the CLI (hkn) worked together.

After a month of attempting to deploy \ac{HAAUKINS} locally, the effort was abandoned. 
The main reasons were the lack of documentation, insufficient information on the interaction between the different components of \ac{HAAUKINS}, 
and the inability to contact the main developer of \ac{HAAUKINS}.
