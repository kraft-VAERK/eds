The project's goal is to shed light on the security risks associated with using pipelines. 
Most \ac{IaaS} providers, such as \ac{GCP}, \ac{AWS}, \ac{Azure}, and others, entail significant costs. 
However, \ac{SDU} is a partner in the recently launched \ac{Ucloud} managed by \ac{DeiC}\cite{deic}, offering educators, professors, 
and students affordable access to real computing power, potentially provided as a grant for research projects.

Despite being an \ac{IaaS} provider \ac{Ucloud} is a little different than traditional 
IaaS providers.
In this section an overview of ucloud as an \ac{IaaS} and emphasizing its characteristics. 
Subsequently, the project aims to delve into the unique aspects of how \ac{Ucloud} functions and outline the distinctions 
between \ac{Ucloud} and other \ac{IaaS} providers. In the following sections, \ac{Ucloud} and its features will be presented as 
how they have been used. In section \ref{sec:discussion-ucloud}, a reflection on the challenges and features that
has been encountered during the project will be discussed.

\subsection{\ac{IaaS}}
\label{sec:iaas}
\ac{IaaS} refers to a cloud computing service wherein the provider hosts 
infrastructure components typically found in an on-premises data center. 
These components include servers, storage, networking hardware, and the virtualization or hypervisor layer. 
The provider also offers various accompanying services, such as detailed billing, monitoring, log access, security features,
load balancing, clustering, and storage resiliency, encompassing backup, replication, and recovery\cite{iaas}.

\ac{Ucloud} bears some resemblance to a conventional \ac{IaaS} provider but exhibits distinct characteristics. 
The primary difference lies in \ac{Ucloud}'s focus on the university environment, 
deviating from a public-oriented approach and the development of traditional applications. 
Instead, \ac{Ucloud} is designed for the utilization of computational power, akin to a regular \ac{IaaS}, 
with a specific emphasis on research and education.

Given its tailored use for application development, 
\ac{Ucloud} does not necessitate certain security measures that would be essential for a more public-facing \ac{IaaS}.

\subsection{Deployment and connection to \ac{Ucloud}}
\label{sec:deployment-and-connetion-ucloud}
\subsubsection{Deployment}
\label{sec:deployment-ucloud}
\ac{Ucloud} offers various sections for deploying virtual machines, which are situated across different universities in Denmark. 
The primary locations utilized are at \ac{SDU} and \ac{AAU}. However, \ac{SDU} instances have a notable drawback as they lack support for nested virtualization, 
rendering them unsuitable for certain applications. This limitation poses an issue, 
especially considering the crucial role of Docker in deploying Gitea and other project components, as discussed further in section \ref{sec:docker}.\\
On the other hand, \ac{AAU} machines are provisioned through the \ac{Ucloud} platform accessible at \url{https://ucloud.sdu.dk/}. 
The process of spawning these machines involves several steps. Firstly, users need to select the desired \ac{OS} and its version. 
Then, they specify a name for the instance and choose the appropriate virtual hardware configuration. Lastly, when deploying a machine, 
users must specify the required hardware or computational power. The hardware configurations are divided into three categories: 
small, medium, and large. These categories differ in the number of \ac{vCPU}, \ac{RAM}, and storage.
\begin{itemize}
    \item Small provides 4 vCPUs, 16 GB RAM, and 100 GB storage.
    \item Medium offers 8 vCPUs, 32 GB RAM, and 100 GB storage.
    \item Large provides 16 vCPUs, 64 GB RAM, and 100 GB storage.
\end{itemize}
This specific selection is specific for certain products available on \ac{Ucloud}, where as 
other product and \ac{VM}s might have different configurations and computational power options.

\subsubsection{Connection}
\label{sec:connection-ucloud}
The connection to the instance relies on \ac{SSH}, with the \ac{SSH} key provided during instance spawning facilitating access. 
However, one significant limitation of this method is the inability to leverage \ac{SSH} connections in a sophisticated manner. 
Other machines on the \ac{Ucloud} network can be accessed through a web client, but all the machines on the 
\ac{AAU} cluster that has the required nested virtualization capabilities are only accessible through \ac{SSH}, without any 
interface to the machine, only \ac{CLI} access.

\subsection{Ucloud interface}
\label{sec:ucloud-interface}
To spawn new machines and interact with the Ucloud infrastructure, you need access to the \ac{Ucloud} interface. 
This access is provided through the wayF login system, a federated single sign-on system used by all Danish universities. 
The wayF login system allows users to log in to multiple systems using the same credentials.

The \ac{Ucloud} UI features multiple sections accessible to users. Upon logging in, the user are greeted by the dashboard.
The dashboard at left hand side of the screen has 5 different menu, which folds out when hovered over. 
As all of them serve a purpose, the options are resources, application, runs and project

\paragraph{Resources}
Here the user has an overview over what is allocated for the project that is selected.
For this project, a grant was applied for and granted. This gave the project access to 
public link, SDU machines resources, AAU machines resources and a project storage.

\paragraph{Applications}
Applications is the section where the user can selection a product that is going to be run.
There is a big variety of machines available for use, but the most used in this project was 
version 22.04 LTS ubuntu for running development and production environments such as the remote Gitea server.

\paragraph{Runs}
Runs the section where the user has an overview of the runs that is running on the currently selected project.
The overview is very simple and gives only information of the run name, status and product in form of the image 
supplied to that particular product.

\paragraph{Projects}
Under the project section, members of a certain project can see and follow the project that they are part of.
\javaf{allocation}, \javaf{members}, \javaf{usage} and so on. It gives a brief over of what is what,
and what are running inside the project.