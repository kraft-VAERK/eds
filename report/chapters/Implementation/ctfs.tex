\section{\ac{CTF}}
\label{sec:ctf}

\subsection{Training repository for Drone}
The knowledge of pipeline configuration is essential in order for a developer to understand how to configure a pipeline.
As there are many pipeline tools and software, the knowledge of the specific pipeline tool is essential.
In this CTF, the user is added to the repository Firelands.
Firelands runs the user through how to construct a pipeline in Drone. The training material comes in five steps:
\begin{itemize}
    \item Step 1 - Pipeline kind, type, name.
    \item Step 2 - Using a steps - Executing a command on a runner.
    \item Step 3 - Using a services, hence a database or a web server.
    \item Step 4 - Using second steps after the first.
    \item Step 5 - Using environment variables and secrets in the steps part.
\end{itemize}

Introducing these parts of the Drone pipeline allows the user to understand and know how 
the pipeline works before delving into the CTF's. A complete solution for the first CTF or training repository can 
be see in \ref{app:firelands-solution}

\subsection{Remote code execution, on a runner}
For pipelines today, execution of code is of great importance. It leaves computational tasks to runners, which normally resides 
on a different machine than the users. That is a good practice, since it leaves the user machine free to focus on quick
development and will not use any computing power on running code. But this also leaves the runner open to attacks.

When using pipelines, the runners execute whatever the pipeline configuration tells it to do without any questions.
In most of the cases, the execution phase is smooth and without any problems, but what if the pipeline is compromised?

\paragraph{Amirdrassil repository\cite{photoview} - Remote code execution CTF's}
In the CTF's a new developer is given access to an already developed project. 
The developer becomes part of the Amirdrassil project 
and is left will all the possibilities to change and execute the pipeline by pushing code to the repository. The developer 
must take control of the pipeline eq. the runners and execute code on the runner to grab the flag. 
The solution to the pipeline can be found in \ref{app:amirdrassil-solution}.

\subsection{Stolen Credentials}
\label{sec:stolen_credentials}
Credentials are the keys to the kingdom. Credentials are essentials to security, since is something really secure if there is no 
way to access, then it might just be classified as inaccessible and not secure. Credentials are a crucial part of a security chain in a 
system. If credentials are stolen, all other security measures are useless.

In todays world of \ac{devops}, accessing remote or internal systems is done by credentials.
Drone handles the credentials either per repository or per organization. Drone does not specify how the secret are stored,
but after a look into the database of Drone, it seems they are stored in clear text in the database. 
it is uncertain whether they are protected by a layer of security measure or not.
However, accessed by the runner with a simple linux command such as \texttt{echo \$\{SECRET\}}, shows the asteriks representation of 
the secret is printed out in the logs.

What this means, if the runner is compromised, the credentials is still safe since the runner does not reveal the secret when printed to stdout.
For the solution to this CTF, the developer takes control of the pipeline. Then changes the pipeline to access secrets stored on the runner and 
print them out in clear text. See solution for the pipeline configuration can be seen in appendix \ref{app:amirdrassil-solution}.

\paragraph{Abberus repository\cite{flask-foundation} - Steal credentials CTF's}
To simulate an attack where an attacker steals secret credentials used to access a registry, consider the following scenario:

The repository simulates a typical organizational repository, with the pipeline defined in the .drone.yml file. 
This pipeline represents the usual steps for executing tasks on either a test branch or the main branch. 
The attacker, positioned as a developer, faces restrictions such as branch protection on the main branch and a repository webhook that triggers only on push events and merge requests related to the main branch.

To steal the credentials, the attacker must navigate these constraints by leveraging their developer access to 
introduce malicious code or configurations within the scope of their permissions, 
thereby exfiltrating the credentials during a legitimate pipeline execution.
\begin{itemize}
    \item First, create a new branch. The name of the challenge is not important.
    \item After the branch has been created, the attacker must change the already constructed pipeline to print out the secrets.
    Drone protects the secrets by changing them to asterisk representation.
    To get the secrets in clear text, the attacker can create a command that changes the secret into another string than the secret.
    \begin{center}
        \javaf{- echo $(echo $PASSWORD | sed 's/./&w/g' | sed 's/w//g')}
    \end{center}
    \item When the attacker has changed the pipeline, the attacker then creates a merge request to the main branch,
    which will trigger the pipeline to run the changed pipeline.
    \item Last the attack copies the changed string to the host pc and removes the extra character that was added.
\end{itemize}

An entire solution of the pipeline can be found in appendix \ref{app:abberus-solution}


\subsection{Supply chain attack}
\label{sec:supply_chain_attack}
A supply chain is when a user of a software program or software developer environment uses
third party, open source software or a other software to build or run their own software or environment.
Meaning that the program is dependent on another program to be bug free, secure, and reliable.

A project might not always use supply chain to run but based on the fact that more companies and developers are more prone to use 
third party software than ever, the risk of a supply chain attack is increasing. 
Information from various sources indicates that the risk of supply chain attacks is increasing. 
This rise is attributed not only to the growing number of third-party providers but also to the increasing number of compromised open-source projects.
\cite{supplychainbrain_supply_chain_breaches}\cite{statista_open_source_supply_chain_attacks}.

A supply chain attack occurs when third-party software is compromised. This software might be a dependency for another application or service.
 An attacker can alter or introduce a vulnerability into the third-party software. 
When the software is updated, the exploitable code becomes accessible, allowing the attacker to target the supposedly \texttt{"secure"} system.

Executing a supply chain attack is no simple feat. Typically, the attacker initiates the process with what is known as an upstream attack. 
In an upstream attack, the assailant targets the software or its provider directly. Once access to the software provider is obtained, 
the attacker may implant a vulnerability within the software. The subsequent phase of the supply chain attack involves the downstream attacker. 
Here, the downstream attack occurs when the newly implanted vulnerability is triggered within software that relies on the compromised software.
\cite{cloudflare_supply_chain_attack}

\paragraph{Ulduar and Icecrown repository - Supply chain attack CTF's}
The supply chain attack consists of two repositories, a repository were the user only has read access and another 
repository where the user has write access. 

\paragraph{Ulduar repository\cite{flask-blog}}
Ulduar is a repository where users have read-only access. It simulates a typical repository containing a pipeline file. 
This pipeline is configured on the drone server to run as a cron job every 2 minutes. 
Although it is uncommon for a pipeline to execute this frequently, it is set this way for simulation purposes to save time. 
The pipeline utilizes a Docker image, 
which is retrieved from the local registry created during the deployment of the entire infrastructure.

\paragraph{Icecrown repository}
In the Icecrown repository, the developer has full access. This repository simulates a supply chain for the Ulduar repository. 
Since the Ulduar repository uses a Docker image from the local registry, which Icecrown can access, it allows the developers to build a new, 
"infected" image within Icecrown's pipeline and push it to the local registry. When the Ulduar repository runs its pipeline, 
it will fetch and execute this new "infected" image. 
The user's task is to alter the Docker image used by the Ulduar repository so that the Ulduar repository prints out its flag.

\paragraph{Solution}
As this \ac{CTF} is not finished due to time constraints, the solution and general
source code is not complete. However the source code for the CTF and pipeline can be found under,\\
\javaf{src/gitea/repositories/Ulduar} and \javaf{src/gitea/repositories/Icecrown}.