Two pipelines using the same programs were developed during this project.
One pipeline with a local Drone instance and a remotely connected Gitea server.
The other is a totally localized deployable pipeline with a local Gitea server and a local Drone instance.
For structural purposes, the pipelines are named \javaf{local} and \javaf{remote}.

\subsection{Local pipeline}
\paragraph{Configuration at initialization}
The local pipeline resolved all the issues encountered with the remote pipeline, 
but it required creating several configuration files for initialization. To ensure the pipeline deployment was fully automated, 
all infrastructure components such as users, repositories, webhooks, and secrets had to be set up before the first login to Gitea and Drone.
Since the setup had to be completed before the actual Gitea and Drone servers were running, manual insertion into the database was necessary. 
This meant that for creating secrets, users, webhooks, and other configurations, 
it was essential to insert the data into the databases of both programs. 
This process often led to large SQLite databases queries with incorrect data entries and unpredictable program behavior.\\
Testing the extensive configuration required running the pipeline and all related programs in Docker containers. 
Each configuration and deployment cycle took four to six minutes. Given that the deployment process for the 
localized pipeline was executed around 300-500 times.
\paragraph{Resource consumption}
As the pipeline used Docker to spawn runners inside a Docker image, the resource consumption 
could potentially be high at a point. Drone gives a restriction in how many runners a pipeline or project 
can run a the same time, but there is no restriction in how many Docker containers a runners can spawn to be able to 
complete its execution. This could potentially lead to a high resource consumption on the host machine.
To prevent this a quota for the runners could be set, but this was not implemented in the pipeline.
Because the quota wasn't implemented into the runners was because a thorough testing of computational resources 
and how much a runner would consume was not done during extreme load. However, it is uncertain whether, if implemented, 
this features should be something done in \ac{HAAUKINS} or something done directly into the pipeline. Implementing
would also mean for local deployment that overusing resources would not be a problem.

\subsection{Remote pipeline}
\paragraph{OAuth2 issue}
The remote pipeline was constructed to address an issue encountered with OAuth2. 
The problem arose because the Drone server and the Gitea server were both running on the same machine using the same URL, 
localhost, differentiated only by port numbers. This caused OAuth2 to malfunction, as it could not distinguish between the two servers, 
resulting in redirects to the wrong server and failed authentication attempts.\\
Through extensive testing and debugging, the solution was found to separate the two servers and assign a 
different URL to the Gitea server. This was achieved by deploying the Gitea server on a remote machine while keeping the drone server 
on a local machine. This allowed OAuth2 to differentiate between the two servers, leading to a successful authentication process.\\
Later in the thesis process, it was discovered that this 
issue could also be resolved on the same machine by using different URLs for the two servers and placing them behind a reverse proxy.
\paragraph{Pipeline structure}
In the remote pipeline setup, the Drone server was localized while the Gitea server was remote. 
This arrangement made the connection and communication between the two servers challenging. 
Normally, both the Drone and Gitea servers would be exposed to the internet. 
However, automating the exposure of the Drone server to the internet was not resolved.\\
To facilitate communication between the two servers, an SSH connection from the Gitea server machine 
to the Drone server was used. This allowed the Gitea server to send \javaf{GET, POST, DELETE, and PUT} requests through 
the SSH connection to the Drone server. This approach was necessary because the Gitea server resided in a VM on \ac{Ucloud}, 
and, as mentioned in section \ref{sec:connection-ucloud}, the VMs on \ac{Ucloud} are only accessible via SSH.
After the \javaf{drone.sh} script was completed, users had to manually create a key pair for the Gitea server, 
enabling it to establish an SSH connection and forward traffic to the Drone server.

Having this structure for the pipeline was not optimal. It had many 
drawbacks and wasn't really an automated deployment. There was almost issues with either 
creating users, webhooks, push repositories to remote Gitea, pipeline not being able to sync with the Gitea server,
Gitea not being able to communicate with the Drone server and so on.
\paragraph{Ngrok issue}
For having the Gitea remotely accessible Ngrok was used to create a public \ac{URL} to the 
server and exposing it. Although Ngrok offers free usage with a single domain, a change in the Terms of Service (TOS) 
limited the amount of data available to free users per month. This limitation rendered the pipeline unusable once 
the data quota was exhausted, making the remote Gitea server inaccessible. 
Consequently, Ngrok was abandoned, and the pipeline was moved to a localized solution.

