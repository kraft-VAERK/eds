\subsection{Drone problems}
\label{sec:discussion-drone}

\subsubsection{Deployment on drone}
The newest version of the Drone server uses Oauth2 to authenticate user account to the server.
This means that another entity, such as gitea authenticates user to the Drone server. This is done by
setting up a Oauth2 application in gitea, and then using the client id and client secret to authenticate
the user to the Drone server. On the Drone official website, it states that it is not recommend to deploy 
Drone with docker-compose, because of the potential network issues that might arise.
\paragraph{Problem 1, integration with gitea over OAuth2}
Other than starting out figuring out how to run Drone, I had to figure out how to integrate it with gitea.
Now later version of Drone, had its own database of users and therefore didn't have to rely on gitea for authentication.
Now this means that before being able to run Drone runners to execute pipeline, the user that wish to use Drone,
needs to authenticate with the gitea server, and post a access token to the gitea server. Essentially the user needs a token 
when is used to authenticate with the gitea server.
Now the authentication process is not a problem, the Drone CI/CD documentation is very specific about how to do it and if done 
correctly it'll work fine.
Now the big problem is posting the token to the gitea server. The gitea server will not accept the token, being posted over http connection,
and it refusing the connection. 
Since Drone need to make it's access token available to the gitea server, it needs to be able to post it to the gitea server.
The problem can be solved by using a proxy or a https connection on the gitea server, but that is not a viable solution.

\paragraph{Problem 2, Advertised problem by Drone}
Now the is linked to the first problem and is about the network aspect of Drone.
The authentication of the user is done in the browser. When that is done Drone obviously needs to be able to communicate with the gitea server.
Now the problem, and im not sure how docker compose works in this part, but what I've deciphered from the internet is that 
Drone might not be able to find the gitea server, and post the token to it.

\paragraph{Problem 3, Drone runner}
When the Drone server is operational, it acts as the intermediary connecting the Git server 
and the runners responsible for executing the pipeline. 
To ensure a smooth pipeline execution, it is imperative that the pipeline configuration 
for the Drone server is exceptionally detailed. This configuration should explicitly define 
crucial details such as the software used to manage the Drone server, particularly Docker if the Drone runners 
are tasked with running the pipeline within Docker containers.

Furthermore, the fact that this specification must be written in YAML introduces a 
level of dependency on YAML syntax and indentation. It's important to note that the 
specific syntax and indentation requirements can vary from one software platform to another when defining the structure of the pipeline. 
Therefore, careful attention to these details is vital to ensure accurate and effective pipeline execution.