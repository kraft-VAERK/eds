As discussed in Section \ref{sec:haaukins}, the absence of documentation and general feedback from the \ac{HAAUKINS} team
made it difficult to fully understand \ac{HAAUKINS} and its capabilities.
The intended use of \ac{HAAUKINS} in this project was to manage the infrastructure for the pipeline backend machines and handle flags.

Discussing the errors and issues that \ac{HAAUKINS} encounters during use is challenging because a 
thorough examination of \ac{HAAUKINS}, its features and structure was not been possible.
Consequently, conclusions about \ac{HAAUKINS} will be based on the information obtained 
from its minimal documentation, the GameSS project, and \ac{DDC}.\\\\
The future use of the \ac{HAAUKINS} program within the pipeline involves deploying 
it on an existing \ac{HAAUKINS} instance through the GameSS project. 
From there, CTFs will be integrated into the new pipeline.

Although the pipeline is not entirely new, it requires a fresh configuration for the \ac{HAAUKINS} platform to correctly spawn the backend containers, 
effectively making it a new pipeline. Additionally, because the \ac{HAAUKINS} platform uses specific DNS records, such as \javaf{hkn}, 
the certificate that enables \ac{HTTPS} communication between containers must be recreated to match the specific \javaf{.hkn} \ac{CN}.