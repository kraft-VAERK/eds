\subsection{\ac{Ucloud} and cloud infrastructure.}
For the pipeline to run together with \ac{HAAUKINS}, a kind of computational power or cloud provider is necessary.
As \ac{SDU} is part of the \ac{Ucloud}\cite{ucloud}, a computational cloud provider, this thesis explored the 
use of \ac{Ucloud} to deploy \ac{HAAUKINS} and the pipeline on that infrastructure.

One of the first problems encountered with \ac{Ucloud} was the allocating of funds. In order 
to spawn and run the desired machines necessary for a project on \ac{Ucloud}, ultimately these machines would have \ac{HAAUKINS} and 
the pipeline running on them.
Applying for grants is something that is done by a professor or a researcher at \ac{SDU}, and as a student,
I had no influence over the allocation of funds.

The next problem was the nested virtualization. Nested virtualization is 
when a operating system gives the user or itself the possibility to run a virtual machine inside a virtual machine. As the 
name suggests, it is a virtualization inside a virtualization.
Nested virtualization is important because, \ac{Ucloud} will spawn 
a virtual machine, on their bare metal which they grant the user access to. Since this project heavily focuses on running Docker images, 
it is crucial to have access to PID 1 on the given machines in order to utilize the Linux systemctl. If the machine is booted without PID 1, 
which is essentially the Linux init system, it will not be able to run Docker images.

Upon starting to work with \ac{Ucloud}, the task of creating an infrastructure and using it publicly, was not as easy as initially thought.
One of the core offerings of an \ac{IaaS} is the ability to run your code and applications on their infrastructure. 
However, Ucloud does not allow users to spawn their own Docker images directly on their infrastructure. Instead, to run Docker images, 
a user must launch an application defined on the Ucloud infrastructure 
and then run the Docker images on that machine, provided the machine has access to PID 1.

Ultimately, the project was completed without using \ac{Ucloud}, and the pipeline was deployed locally. 
The integration of \ac{Ucloud} with \ac{HAAUKINS} and a pipeline remains uncertain. Out from the gathered 
material and information about \ac{Ucloud} it seems, that it is only feasible to utilize \ac{Ucloud} 
if \ac{DeiC} modifies its restrictions on internet access for spawned machines, 
allowing bidirectional traffic to and from machines on the \ac{Ucloud} infrastructure on other protocols than \ac{SSH}.